%%%%%%%%%%%%%%%%%%%%%%%%%%%%%%%%%%%%%%%%%
% Cleese Assignment (For Students)
% LaTeX Template
% Version 2.0 (27/5/2018)
%
% This template originates from:
% http://www.LaTeXTemplates.com
%
% Author:
% Vel (vel@LaTeXTemplates.com)
%
% License:
% CC BY-NC-SA 3.0 (http://creativecommons.org/licenses/by-nc-sa/3.0/)
% 
%%%%%%%%%%%%%%%%%%%%%%%%%%%%%%%%%%%%%%%%%

%----------------------------------------------------------------------------------------
%	PACKAGES AND OTHER DOCUMENT CONFIGURATIONS
%----------------------------------------------------------------------------------------

\documentclass[10pt]{article}
\input{activite.sty} % Include the file specifying the document structure and custom commands
% \usepackage{activite}
%----------------------------------------------------------------------------------------
%	ASSIGNMENT INFORMATION
%----------------------------------------------------------------------------------------

% Required
\newcommand{\assignmentQuestionName}{Question} % The word to be used as a prefix to question numbers; example alternatives: Problem, Exercise
\newcommand{\assignmentClass}{Physique Chimie} % Course/class
\newcommand{\assignmentTitle}{Activité n°3} % Assignment title or name
\newcommand{\assignmentAuthorName}{Chapitre 3} %

%----------------------------------------------------------------------------------------
%	VARIABLES
%----------------------------------------------------------------------------------------

\newcommand{\titreActivite}{Activité 3: Élaborer un circuit électrique.} % titre de l'activité
\newcommand{\objectif}{ 	
	
	\begin{itemize}
		\item Élaborer et réaliser un circuit électrique répondant à un cahier des charges simple.
	\end{itemize}
} % titre de l'activité
\newcommand{\contexte}{
	L’enseignant.e de physique chimie veut réaliser un circuit permettant 
	d’informer les élèves de la procédure à suivre en cas de retard.
	\begin{itemize}
		\item 	Il souhaite installer un feu tricolore sur la porte de sa salle:
		\begin{itemize}
			\item vert : l’élève peut rentrer sans frapper.
			\item orange: l’élève doit frapper à la porte puis s’excuser 
			de son retard mais il peut encore assister au cours.
			\item rouge : l’élève doit descendre à la vie scolaire.
		\end{itemize}
		\item Il veut aussi des lampes témoins sur son bureau, 
		pour connaître quel lampe est allumée à l'extérieure.
		\item Chaque paire de lampes doit fonctionner indépendamment et chaque lampe témoin ne doit s’allumer que si la lampe
		qui correspond dehors est allumée.		
	\end{itemize}
	}
\newcommand{\resumeContexte}{
	Saura-tu aider l'eneignant.e de physique chimie 
	à \emph{enfin avoir la paix, bon sang?!} (et des lampes sur sa porte.)
} % titre de l'activité




%----------------------------------------------------------------------------------------

\begin{document}
%----------------------------------------------------------------------------------------
%	TITLE PAGE
%----------------------------------------------------------------------------------------
\date{}
\title{\titreActivite}
\maketitle % Print the title page

%----------------------------------------------------------------------------------------
%	QUESTION 1
%----------------------------------------------------------------------------------------

\underline{\textbf{Objectif}} :  \vspace{2pt}
\objectif

\vspace{4pt}

\underline{\textbf{Contexte}} :  \textit{\contexte}

\begin{center}
	\includegraphics[width=0.25\columnwidth]{activite.jpg} % Example image
\end{center}

\textbf{\resumeContexte}
\vspace{-12pt}
%% ----------------------------------

\assignmentSection{Votre mission travail}
%----------------------------------------------------------------------------------------
%	QUESTION 
%----------------------------------------------------------------------------------------

\begin{question}
	\questiontext{
		En groupe de 3 ou 4 : \textbf{Réaliser} sur le cahier ou au dos de l'activité
		le schéma du circuit électrique qui permet de 
		satisfaire l'enseignant.e.	
	}
	\end{question}
	\subquestion{\textbf{Faire valider.}}
	\subquestion{\textbf{Réaliser le circuit.}}
	\subquestion{\textbf{Faire valider.}}


%----------------------------------------------------------------------------------------
%	QUESTION 
%----------------------------------------------------------------------------------------

\begin{question}
	\questiontext{
		L'enseigant.e est très content.e de son nouveau circuit. 
		Mais il.elle a une nouvelle commande, cette fois-ci : \begin{itemize}
			\item Il veut un feu vert et un feu rouge à sa porte(avec un témoin vert et un rouge à son bureau)
			\item Il veut un interrupteur qui lui permet de tout éteindre.
		\end{itemize}
		
		}
	\end{question}
%	\answerbox{}


\end{document}
