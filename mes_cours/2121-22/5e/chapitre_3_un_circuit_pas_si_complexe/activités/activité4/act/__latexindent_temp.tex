%%%%%%%%%%%%%%%%%%%%%%%%%%%%%%%%%%%%%%%%%
% Cleese Assignment (For Students)
% LaTeX Template
% Version 2.0 (27/5/2018)
%
% This template originates from:
% http://www.LaTeXTemplates.com
%
% Author:
% Vel (vel@LaTeXTemplates.com)
%
% License:
% CC BY-NC-SA 3.0 (http://creativecommons.org/licenses/by-nc-sa/3.0/)
% 
%%%%%%%%%%%%%%%%%%%%%%%%%%%%%%%%%%%%%%%%%

%----------------------------------------------------------------------------------------
%	PACKAGES AND OTHER DOCUMENT CONFIGURATIONS
%----------------------------------------------------------------------------------------

\documentclass[10pt]{article}
\input{activite.sty} % Include the file specifying the document structure and custom commands
% \usepackage{activite}
%----------------------------------------------------------------------------------------
%	ASSIGNMENT INFORMATION
%----------------------------------------------------------------------------------------

% Required
\newcommand{\assignmentQuestionName}{Question} % The word to be used as a prefix to question numbers; example alternatives: Problem, Exercise
\newcommand{\assignmentClass}{Physique Chimie} % Course/class
\newcommand{\assignmentTitle}{Activité n°4} % Assignment title or name
\newcommand{\assignmentAuthorName}{Chapitre 3} %

%----------------------------------------------------------------------------------------
%	VARIABLES
%----------------------------------------------------------------------------------------

\newcommand{\titreActivite}{\huge Activité 4: Dominos et évaluation expérimentale \vspace{-10pt}} % titre de l'activité
\newcommand{\objectif}{ 	
	
	\begin{itemize}
		\item Se réapproprier et consolider les connaissances 
		et compétences d'électricité vu jusqu'ici.
	\end{itemize}
} 

\newcommand{\myboxtwo}{\makebox(40,15){}}

%----------------------------------------------------------------------------------------

\begin{document}
%----------------------------------------------------------------------------------------
%	TITLE PAGE
%----------------------------------------------------------------------------------------
\date{}
\title{\titreActivite}
\maketitle % Print the title page

\large \vspace{10pt}
\underline{Consignes générale :}
\begin{itemize}
	\item La classe est séparée en deux groupes qui vont chacun 
	commencer par un travail différent puis alterner.
	\item La note de l'évaluation expérimentale est donnée à la fin
	 de la séance (la priorité est l'évaluation expérimentale).
	\item La note des jeux de dominos \textbf{est un bonus, non prioritaire}  
	et cette notes sera combiné à d'autres exercices futurs.
\end{itemize}
\vspace{-20pt}
\assignmentSection{Travail 1: Jeux de dominos d'électricité.}
\vspace{-10pt}
	\begin{itemize}
		\item Dans ce dominos d'électricité, \textbf{les cartes sont séparés en deux}. 
		Elles possèdent une partie schéma normalisés  et une partie dessin qui ne se correspondent pas. 
		\item La règle à respecter est qu'il faut mettre les cartes les unes à 
		côté des autres pour que \textbf{chaque dessin soit à côté de son schéma}. 
		\item À la fin, \textbf{les cartes doivent former un carré} (aucune carte laissée de côté).
		\item Il faut \textbf{commencer par le niveau 1} (aller chercher le jeux sur la table dominos).
		\item Quand on a terminé, 
		\begin{itemize}
			\item \textbf{Écrire l'ordre dans lequel on a disposé les cartes} (grâce à leurs numéros).
			\item \textbf{Demander} à l'enseignant.e de vérifier.
			\item \textbf{Ranger les cartes} avec l'élastique et \textbf{Passer} au niveau supérieur.  
		\end{itemize}	
		  
	\end{itemize}

	\begin{center}	
		\begin{tabular}{@{}|c|c|c|c|c|@{}} \toprule
			\multicolumn{5}{|c|}{Note} \\ \midrule
			\myboxtwo & \myboxtwo & \myboxtwo & \myboxtwo & \myboxtwo \\ \midrule
			1 & 1.5 & 2 & 2 & 3 \\ \bottomrule
		\end{tabular}
	\end{center}
\vspace{-20pt}
\assignmentSection{Travail 2: Évaluation expérimentale.}
\vspace{-10pt}
		\begin{itemize}
			\item Par groupe de trois ou quatre, \textbf{Réaliser} chacun des huit circuits dans l'ordre.
			\item Quand un circuit est terminé, il faut appeler l'enseignant.e pour valider.
		\end{itemize}
\begin{center}	
	\begin{minipage}[c]{0.2\textwidth}
		\centering \includegraphics[width=0.8\columnwidth]{circuit1.png} circuit 1
	\end{minipage}
	\hspace{ 0pt}
	\begin{minipage}[c]{0.2\textwidth}
		\centering \includegraphics[width=0.8\columnwidth]{circuit2.png} circuit 2
	\end{minipage}
	\hspace{ 0pt}
	\begin{minipage}[c]{0.2\textwidth}
		\centering \includegraphics[width=0.8\columnwidth]{circuit3.png} circuit 3
	\end{minipage}
	\hspace{ 0pt}
	\begin{minipage}[c]{0.2\textwidth}
		\centering \includegraphics[width=0.8\columnwidth]{circuit4.png} circuit 4
	\end{minipage}
	\hspace{ 0pt}

	\begin{minipage}[c]{0.2\textwidth}
		\centering \includegraphics[width=0.8\columnwidth]{circuit5.png} circuit 5
	\end{minipage}
	\hspace{ 0pt}
	\begin{minipage}[c]{0.2\textwidth}
		\centering \includegraphics[width=0.8\columnwidth]{circuit6.png} circuit 6
	\end{minipage}
	\hspace{ 0pt}
	\begin{minipage}[c]{0.2\textwidth}
		\centering \includegraphics[width=0.8\columnwidth]{circuit7.png} circuit 7
	\end{minipage}
	\hspace{ 0pt}
	\begin{minipage}[c]{0.2\textwidth}
		\centering \includegraphics[width=0.7\columnwidth]{circuit8.png} circuit 8
	\end{minipage}
	\hspace{ 0pt}
\end{center}

\begin{center}	
		\begin{tabular}{@{}|c|c|c|c|c|c|c|c|@{}} \toprule
			\multicolumn{8}{|c|}{Note} \\ \midrule
			\myboxtwo & \myboxtwo & \myboxtwo & \myboxtwo & \myboxtwo & \myboxtwo & \myboxtwo & \myboxtwo \\ \midrule
			1& 1& 1 & 1.5 & 1.5 & 1.5 & 1.5 & 1.5  \\ \bottomrule
		\end{tabular}
\end{center}

\end{document}
