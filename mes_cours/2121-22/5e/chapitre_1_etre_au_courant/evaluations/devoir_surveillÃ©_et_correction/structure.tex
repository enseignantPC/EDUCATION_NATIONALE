%----------------------------------------------------------------------------------------
%	PACKAGES AND OTHER DOCUMENT CONFIGURATIONS
%----------------------------------------------------------------------------------------

\usepackage{lastpage} % Required to determine the last page number for the footer

\usepackage[utf8]{inputenc}
\usepackage[T1]{fontenc}
\usepackage[french]{babel}      %Originally for french document, change to english or relevant language

\usepackage{amsmath,amssymb}
\usepackage{multicol}
\usepackage[dvipsnames,svgnames]{xcolor}
\usepackage{tikz}
\usetikzlibrary{fadings}
\usetikzlibrary{calc}

\usepackage{tkz-tab}
\usepackage{pgfplots}
\pgfplotsset{compat=1.17}


\usepackage{multido} % required for answerbox
\usepackage[most]{tcolorbox} % Required for boxes that split across pages

%----------------------------------------------------------------------------------------
%	MARGINS
%----------------------------------------------------------------------------------------

\usepackage{geometry} % Required for adjusting page dimensions and margins

\geometry{
	paper=a4paper, % Change to letterpaper for US letter
	top=1.5cm, % Top margin
	bottom=2.5cm, % Bottom margin
	left= 1cm, % Left margin
	right=1cm, % Right margin
	headheight=14pt, % Header height
	footskip=1.4cm, % Space from the bottom margin to the baseline of the footer
	headsep=14pt, % Space from the top margin to the baseline of the header
	%showframe, % Uncomment to show how the type block is set on the page
}

%----------------------------------------------------------------------------------------
%	FONT
%----------------------------------------------------------------------------------------
\usepackage{fontspec}
\usepackage{unicode-math}
\usepackage[utf8]{inputenc} % Required for inputting international characters
\usepackage[T1]{fontenc} % Output font encoding for international characters

% \usepackage[sfdefault,light]{roboto} % Use the Roboto font
\setmainfont[Ligatures=TeX]{Caladea}

%----------------------------------------------------------------------------------------
%	HEADERS AND FOOTERS
%----------------------------------------------------------------------------------------


%Format Header and footer
\pagestyle{headandfoot}
\header{Physique Chimie}{\Large\textbf{Chapitre 1}}{
	\footnotesize  
	Classe : \ldots\ldots\ldots\ldots\ldots\ldots
	\\ Nom et prénom: \ldots\ldots\ldots\ldots\ldots\ldots\ldots\ldots\ldots\ldots\ldots\ldots\ldots\ldots\ldots\ldots\ldots\ldots 
}
\headrule
\footrule
\setlength{\columnsep}{0.25cm}
\setlength{\columnseprule}{1pt}
\footer{}{ Page \thepage \hphantom s sur \pageref{LastPage} }{}
% \extrafootheight{-2cm}

\pagestyle{plain}

% Change section command behavior
\usepackage{titlesec}
% \titleformat{\section}[frame]{\Huge\bfseries\filright}{}{2mm}{\centering Devoir surveillé: }

\titleformat
{\section} % Section type being modified
[frame] % Shape type, can be: hang, block, display, runin, leftmargin, rightmargin, drop, wrap, frame
{\Huge\bfseries\filright} % Format of the whole section
{\thesection} % Format of the section label
{6pt} % Space between the title and label
{\centering Devoir surveillé: } % Code before the label


\usepackage{titling} % Allows custom title configuration



\pretitle{
	\vspace{200pt} % Move the entire title section up
}


% \titlespacing{\section}{0pt}{0.5\baselineskip}{0.5\baselineskip} % Spacing around section titles, the order is: left, before and after



% Add a watermark if answers are shown
\ifprintanswers
\usepackage{draftwatermark}
\SetWatermarkColor{red!30}
\SetWatermarkScale{1.1}
\SetWatermarkText{Correction}     %Watermark text
\fi



% contains the answer to a question or a box where the student can reply
% #1 number of lines to reply
% #2 answer to the question, centered
%  old
\newcommand{\answerbox}[2]{
	\ifprintanswers
		\begin{center}
			\color{DarkRed}
			\textbf{\textit{#2}}
		\end{center}

	\else
	\begin{tcolorbox}[breakable, enhanced]
		% \vphantom{L}
		% \vspace{\numexpr #1-1\relax\baselineskip} % \vphantom{L} to provide a typesetting strut with a height for the line, \numexpr to subtract user input by 1 to make it 0-based as this command is
		\vspace{15pt}
		\multido{}{#1}{\noindent\makebox[\linewidth]{\dotfill}
		\endgraf\vspace{12pt}}% ... dotted lines ...
		\vspace{-13pt}
	\end{tcolorbox}
	\fi
}
% new
% \newcommand{\answerbox}[1]{
% 	\begin{tcolorbox}[breakable, enhanced]
% 		% \vphantom{L}
% 		% \vspace{\numexpr #1-1\relax\baselineskip} % \vphantom{L} to provide a typesetting strut with a height for the line, \numexpr to subtract user input by 1 to make it 0-based as this command is
% 		\vspace{15pt}
% 		\multido{}{#1}{\noindent\makebox[\linewidth]{\dotfill}
% 		\endgraf\vspace{12pt}}% ... dotted lines ...
% 		\vspace{-13pt}
% 	\end{tcolorbox}
% }


\renewcommand*\half{.5}

% \qformat{\textbf{\large Question \thequestion\hfill}  (\thepoints)}