%%%%%%%%%%%%%%%%%%%%%%%%%%%%%%%%%%%%%%%%%
% Beamer Presentation
% LaTeX Template
% Version 1.0 (10/11/12)
%
% This template has been downloaded from:
% http://www.LaTeXTemplates.com
%
% License:
% CC BY-NC-SA 3.0 (http://creativecommons.org/licenses/by-nc-sa/3.0/)
%
%%%%%%%%%%%%%%%%%%%%%%%%%%%%%%%%%%%%%%%%%

%----------------------------------------------------------------------------------------
%	PACKAGES AND THEMES
%----------------------------------------------------------------------------------------

\documentclass{beamer}

\mode<presentation> {

% The Beamer class comes with a number of default slide themes
% which change the colors and layouts of slides. Below this is a list
% of all the themes, uncomment each in turn to see what they look like.


% \usetheme{default}
% \usetheme{AnnArbor}
% \usetheme{Antibes}
% \usetheme{Bergen}
% \usetheme{Berkeley} % neat
% \usetheme{Berlin} % nice
% \usetheme{Boadilla}
% \usetheme{CambridgeUS} %nice colors
% \usetheme{Copenhagen} % neat
% \usetheme{Darmstadt} % simple and neat
% \usetheme{Dresden} %nice and neat
% \usetheme{Frankfurt} % clean and -neat
% \usetheme{Goettingen} % clean and -neat
% \usetheme{Hannover} % neat and clean
% \usetheme{Ilmenau} % nice but dangerous layout
% \usetheme{JuanLesPins} % nice
% \usetheme{Luebeck} % clean and nice
% \usetheme{Malmoe} % same
% \usetheme{Madrid} % simple
\usetheme{Marburg} % beautiful simple and neat
% \usetheme{Montpellier} % very simple
% \usetheme{PaloAlto} % clear and nice and neat
% \usetheme{Pittsburgh}
% \usetheme{Rochester} %simple and not neat 
% \usetheme{Singapore} %meh
% \usetheme{Szeged} % a bit ugly 
% \usetheme{Warsaw} % clear and neat

% As well as themes, the Beamer class has a number of color themes
% for any slide theme. Uncomment each of these in turn to see how it
% changes the colors of your current slide theme.

% \usecolortheme{albatross}
% \usecolortheme{beaver}
% \usecolortheme{beetle}
% \usecolortheme{crane} % nice
\usecolortheme{dolphin} % very nice
% \usecolortheme{dove}
% \usecolortheme{fly}
% \usecolortheme{lily} % clean
% \usecolortheme{orchid} % clean
% \usecolortheme{rose}
% \usecolortheme{seagull}
% \usecolortheme{seahorse}
% \usecolortheme{whale} % clean
% \usecolortheme{wolverine}

% \setbeamertemplate{footline} % To remove the footer line in all slides uncomment this line
\setbeamertemplate{footline}[page number] % To replace the footer line in all slides with a simple slide count uncomment this line

\setbeamertemplate{navigation symbols}{} % To remove the navigation symbols from the bottom of all slides uncomment this line
}

\usepackage{graphicx} % Allows including images
\usepackage{booktabs} % Allows the use of \toprule, \midrule and \bottomrule in tables

%----------------------------------------------------------------------------------------
%	TITLE PAGE
%----------------------------------------------------------------------------------------

\title[Présentation]{Séance 2} % The short title appears at the bottom of every slide, the full title is only on the title page

\author{A. Cercy} % Your name
\institute[] % Your institution as it will appear on the bottom of every slide, may be shorthand to save space
{
Collège Pablo Neruda
}
% \date{\today} % Date, can be changed to a custom date
\date{} % Date, can be changed to a custom date

\begin{document}

\begin{frame}
\titlepage % Print the title page as the first slide
\end{frame}

\begin{frame}
\frametitle{Plan} % Table of contents slide, comment this block out to remove it
\tableofcontents % Throughout your presentation, if you choose to use \section{} and \subsection{} commands, these will automatically be printed on this slide as an overview of your presentation
\end{frame}
%%-------------------------------------------------------------

% \begin{frame}
%     \frametitle{Pour commencer}
%     \centerline{Se mettre à sa place habituelle et rester debout.}
% \end{frame}


% \section{Nouveau chapitre}
% \begin{frame}
% \frametitle{Nouveau chapitre}
% Prendre la page blanche suivante du cahier et tout en haut, en rouge et souligné, écrire "Chapitre 1 : Être au courant" puis \textbf{laisser le reste de la page blanche}.
% \end{frame}


% \section{L'activité du jour}
% \begin{frame}
% \frametitle{L'activité du jour}
% Sur la page suivante, coller l'activité 1 : Perdus dans le noir
% \end{frame}


% \section{Lecture de l'activité}
% \begin{frame}
% \frametitle{Lecture de l'activité}
% Nous lisons l'activité ensemble.
% \end{frame}


% \section{Matériel et question 1}
% \begin{frame}
% \frametitle{Matériel et question 1}
% Question 1 : Par binôme, réaliser un circuit électrique pour allumer l’ampoule. Vous disposez d’une
% pile cylindrique, d’un fil et d’une ampoule...
% \end{frame}


% \section{Question 2}
% \begin{frame}
% \frametitle{Question 2}

% (a) Dessinez tous les circuits possibles où la lampe s’allume.
% Il faut qu'on voit très précisément l'ampoule et la pile et leur différentes parties.

% \vspace{10pt}
% \textbf{\textit{Avant les dessins, écrivez "Question 2 (a)" sur votre cahier:}}

% \end{frame}



% \section{Question 2 suite}
% \begin{frame}
% \frametitle{Question 2 suite}
% (b) Même travail mais avec une pile plate (qu’il faut aller chercher) qui remplace la pile cylindrique.

% \vspace{5pt}
% (c) Dessinez  une ampoule avec le filament à l’intérieur de l’ampoule.
% \end{frame}



% \section{Question 2 suite}
% \begin{frame}
% \frametitle{Question 2 suite}
% Question 2 :

% (d)Vous venez de réaliser un circuit électrique pour allumer la lampe ! 

% Quelles sont les conditions pour que la lampe s’allume, autrement dit, comment définiriez vous un circuit électrique ?
% \end{frame}


% \section{Correction}
% \begin{frame}
%     \frametitle{Correction}
%     Il faut participer à la correction et la recopier sur l'activité.
% \end{frame}

% \section{Cours}
%     \begin{frame}
%     \frametitle{Cours}
%     Il faut recopier le cours dans \textbf{\textit{En dessous du début du chapitre}}
    
% \end{frame}


\begin{frame}
\Huge{\centerline{Fin}}
\end{frame}

%----------------------------------------------------------------------------------------

\end{document} 