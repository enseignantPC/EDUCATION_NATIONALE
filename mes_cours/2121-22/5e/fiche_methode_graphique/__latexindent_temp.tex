%%%%%%%%%%%%%%%%%%%%%%%%%%%%%%%%%%%%%%%%%
% Cleese Assignment (For Students)
% LaTeX Template
% Version 2.0 (27/5/2018)
%
% This template originates from:
% http://www.LaTeXTemplates.com
%
% Author:
% Vel (vel@LaTeXTemplates.com)
%
% License:
% CC BY-NC-SA 3.0 (http://creativecommons.org/licenses/by-nc-sa/3.0/)
% 
%%%%%%%%%%%%%%%%%%%%%%%%%%%%%%%%%%%%%%%%%

%----------------------------------------------------------------------------------------
%	PACKAGES AND OTHER DOCUMENT CONFIGURATIONS
%----------------------------------------------------------------------------------------



\documentclass[10pt]{article}

%%%%%%%%%%%%%%%%%%%%%%%%%%%%%%%%%%%%%%%%%
% Cleese Assignment
% Structure Specification File
% Version 1.0 (27/5/2018)
%
% This template originates from:
% http://www.LaTeXTemplates.com
%
% Author:
% Vel (vel@LaTeXTemplates.com)
%
% License:
% CC BY-NC-SA 3.0 (http://creativecommons.org/licenses/by-nc-sa/3.0/)
% 
%%%%%%%%%%%%%%%%%%%%%%%%%%%%%%%%%%%%%%%%%

%----------------------------------------------------------------------------------------
%	PACKAGES AND OTHER DOCUMENT CONFIGURATIONS
%----------------------------------------------------------------------------------------

\usepackage{lastpage} % Required to determine the last page number for the footer

\usepackage{graphicx} % Required to insert images

\setlength\parindent{0pt} % Removes all indentation from paragraphs

\usepackage[svgnames]{xcolor} % Enabling colors by their 'svgnames'
\usepackage[most]{tcolorbox} % Required for boxes that split across pages

\usepackage{booktabs} % Required for better horizontal rules in tables

\usepackage{listings} % Required for insertion of code

\usepackage{etoolbox} % Required for if statements

\usepackage{multido} % required for dotted lines

\usepackage[french]{babel} % English language hyphenation

\usepackage{svg}

\usepackage{wrapfig}

%----------------------------------------------------------------------------------------
%	MARGINS
%----------------------------------------------------------------------------------------

\usepackage{geometry} % Required for adjusting page dimensions and margins

\geometry{
	paper=a4paper, % Change to letterpaper for US letter
	top=2cm, % Top margin
	bottom=2cm, % Bottom margin
	left = 1.5cm, % Left margin
	right= 1.5cm, % Right margin
	headheight=14pt, % Header height
	footskip=1.4cm, % Space from the bottom margin to the baseline of the footer
	headsep=1.2cm, % Space from the top margin to the baseline of the header
	%showframe, % Uncomment to show how the type block is set on the page
}

%----------------------------------------------------------------------------------------
%	FONT
%----------------------------------------------------------------------------------------
\usepackage{fontspec}
\usepackage{unicode-math}
\usepackage[utf8]{inputenc} % Required for inputting international characters
\usepackage[T1]{fontenc} % Output font encoding for international characters

\setmainfont[Ligatures=TeX]{Caladea}
%----------------------------------------------------------------------------------------
%	HEADERS AND FOOTERS
%----------------------------------------------------------------------------------------




\tcbset{
	colback=blue!10!white,
	colframe=orange!75!red!75!black,
	% equal height group=AT,
	before=,
	after=\hfill,
	fonttitle=\bfseries,
	subtitle style={
		halign = center,
		boxrule=0.4pt,
		colback=orange!75!red!75!black!90!orange,
		attach title to upper={\ --- \ }
		},
	}



\newtcolorbox{titlebox}[2][]{
    colbacktitle=red!85!black,
    enhanced,
    attach boxed title to top center={yshift=-2mm},
    boxrule=1mm,
	title code app={\path[tcb fill title]
	(title) circle (5mm); },
	title={\Huge \textbf{#2}},#1}



\newtcolorbox{mybox}[1][]{
    colbacktitle=red!10!white,
    colback=blue!10!white,
    coltitle=red!70!black,
    adjusted title={#1},
    fonttitle=\bfseries,
    attach title to upper={\ --- \ }
    }



\usepackage{fancyhdr} % Required for customising headers and footers

% \lhead{\tcbincludegraphics[width=3cm,colback=white]{electricien_fou.jpeg}}
\renewcommand{\headrulewidth}{0pt}
 % Include the file specifying the document structure and custom commands
\usepackage{tikz}
\usepackage{svg}

%----------------------------------------------------------------------------------------
%	VARIABLES
%----------------------------------------------------------------------------------------

\newcommand{\titre}{Fiche méthode : Tracer des graphiques} % titre de l'activité
\newcommand{\introduction}{
    Lorsque les scientifiques font des expériences, ils.elles
    remplissent généralement des 
    \textbf{\color{DarkRed} {tableaux de donnée}}.
    
    Par exemple, une physicienne peut faire chauffer de l'eau et noter 
    régulièrement la température de l'eau. Il y aura alors deux données importantes : 
    La \textbf{\color{DarkRed} {température}} et 
    \textbf{\color{DarkRed} {l'intervalle de temps qui est passé}}.

    Pour \textbf{\color{DarkRed} {visualiser des données}}, on va utiliser un 
    \textbf{\color{DarkRed} {graphique}} (ou courbe). Il s'agit d'un des outils 
    les plus puissants des scientifiques et maîtriser sa puissance 
    prendra des années.
}

%----------------------------------------------------------------------------------------

\begin{document}
\tikzstyle{every picture}+=[remember picture]
\tikzstyle{na} = [shape=rectangle,inner sep=0pt,text depth=0pt]

\thispagestyle{fancy}
\cfoot{}


\begin{titlebox}{\titre}
    \tcbsubtitle{introduction:}    
    % for wrapping an icon left or right ----------------------------
    % \vspace{10pt}

    % \begin{wrapfigure}[]{L}{3.5cm}
    %     \vspace*{-0.6cm}
    %     \tcbincludegraphics[width=3cm,colback=white]{icon.png}
    % \end{wrapfigure}%
    
    \hspace{0pt}
    \begin{minipage}[c]{0.15\textwidth}
        \tcbincludegraphics[width=\columnwidth]{icon.png}
    \end{minipage}
    \hspace{20pt}
    \begin{minipage}[c]{0.8\textwidth}
        \textit{\introduction}
    \end{minipage}
    \hspace{-20pt}
    % ---------------------------------------------------------------
    
    \tcbsubtitle{1. Je trace les axes.} 

    \underline{Exemple d'énoncé:}

    \begin{center}
        \textit{« Trace la courbe qui représente l'évolution de la \tikz\node[na](word1){température}; en fonction du \tikz\node[na](word2){temps};. »}
    \end{center}
    
    \begin{minipage}[t]{0.5\textwidth}
        \centering Dans ce cas là, \tikz\node[na](word3){l'axe des ordonnées}; (vertical) représente la température.
    \end{minipage}
    \hspace{5pt}
    \begin{minipage}[c]{0.5\textwidth}
        \centering \tikz\node[na](word4){L'axe des abscisses}; (horizontal) représente le temps.
    \end{minipage}
      
    \begin{tikzpicture}[overlay]
        \path[<->,red,thick](word3) edge [out=20, in=-20] (word1);
        \path[<->,red,thick](word4) edge [out=20, in=-20] (word2);
    \end{tikzpicture}

    \vspace{5pt}

    \begin{minipage}[l]{0.45\textwidth}
        \centering
        Je trace deux axes perpendiculaires, sur lesquels j’indique :
    \begin{itemize}
        \item En abscisse (horizontalement) le nom de la variable connue (ici le temps) et son unité (ici la seconde).
        \item En ordonnée (verticalement) le nom de la variable mesurée (ici la température) et son unité (ici le degré Celsius : °C)
        \item Je gradue les axes en indiquant quelques valeurs :
        par exemple 0, 5, 10 en abscisse et 0, 10, 20, 30 en ordonnée.
    \end{itemize}
    \end{minipage}
    \hspace{20pt}
    \begin{minipage}[c]{0.45\textwidth}
        \centering
        \tcbincludegraphics[width=0.9\columnwidth]{clipped.png}
        \hspace{-10pt}
    \end{minipage}
    
    \tcbsubtitle{2. Je choisis l'échelle.} 
    Je choisis une échelle et je l’indique clairement sur mon graphique. Par exemple :
    \begin{itemize}
        \item 1 cm représente 5°C se note : \framebox{1cm \leftrightarrow 5°C}
        \item 1 cm représente 10 secondes se note : \framebox{1cm \leftrightarrow 10 sec}
    \end{itemize}

    On peut aussi choisir \framebox{1 carreau \leftrightarrow 5°C}
    \tcbsubtitle{3. Je trace les points.} 
    Les points sont représentées par des croix ( + ) placées à l’intersection de la ligne verticale passant par l’abscisse et de la ligne
    horizontale passant par l’ordonnée.

    % for wrapping an icon left or right ----------------------------
        % \begin{wrapfigure}[]{R}{2.8cm}
        %     \vspace*{-0.75cm}
        %     \tcbincludegraphics[width=3cm,colback=white]{icon.jpeg}
        % \vspace{-30pt}
        % \end{wrapfigure}%
        % % ---------------------------------------------------------------
        % \phantom{0pt}
        % \vspace{4cm}
    % \linebreak[4]
    % \tcblower
    \tcbsubtitle{4. Je dessines la courbe.} 
    Elle doit être régulière et doit passer le plus près possible des croix mais on ne relie pas tous les points par des segments.

    \tcbsubtitle{5. J'écris le titre du graphique} 
    \begin{center}        
    \begin{minipage}[c]{0.25\textwidth}
        Pour terminer, on donne un titre au graphique 
        (ici, « Evolution de la température en fonction du temps »
        convient).

        Après avoir suivi toutes les étapes, un graphique peut
        ressembler à ça:
    \end{minipage}
    \hspace{10pt}
    \begin{minipage}[c]{0.7\textwidth}
        \centering
        \tcbincludegraphics[width=0.9\columnwidth]{fg.png}
    \end{minipage}
    \end{center}
    
\end{titlebox}


\end{document}
