%%%%%%%%%%%%%%%%%%%%%%%%%%%%%%%%%%%%%%%%%
% Cleese Assignment (For Students)
% LaTeX Template
% Version 2.0 (27/5/2018)
%
% This template originates from:
% http://www.LaTeXTemplates.com
%
% Author:
% Vel (vel@LaTeXTemplates.com)
%
% License:
% CC BY-NC-SA 3.0 (http://creativecommons.org/licenses/by-nc-sa/3.0/)
% 
%%%%%%%%%%%%%%%%%%%%%%%%%%%%%%%%%%%%%%%%%

%----------------------------------------------------------------------------------------
%	PACKAGES AND OTHER DOCUMENT CONFIGURATIONS
%----------------------------------------------------------------------------------------

\documentclass[10pt]{article}
\input{activite.sty} % Include the file specifying the document structure and custom commands
% \usepackage{activite}
%----------------------------------------------------------------------------------------
%	ASSIGNMENT INFORMATION
%----------------------------------------------------------------------------------------

% Required
\newcommand{\assignmentQuestionName}{Question} % The word to be used as a prefix to question numbers; example alternatives: Problem, Exercise
\newcommand{\assignmentClass}{Physique Chimie} % Course/class
\newcommand{\assignmentTitle}{Activité n°3} % Assignment title or name
\newcommand{\assignmentAuthorName}{Chapitre 2} %

%----------------------------------------------------------------------------------------
%	VARIABLES
%----------------------------------------------------------------------------------------

\newcommand{\titreActivite}{Activité 3 : Le court circuit} % titre de l'activité
\newcommand{\objectif}{ 	
	
	\begin{itemize}
		\item Danger du court-circuit et impact sur les autres dipôles.
		\item Trajet du courant électrique dans un circuit avec court-circuit.
		\item Rôle du cache-prise, du fusible et du disjoncteur.
		\item Danger et précautions liés à l'électrisation et l'électrocution.
	\end{itemize}
} % titre de l'activité
\newcommand{\contexte}{
	% Laeticia raconte ...
}
\newcommand{\resumeContexte}{
	% Saura-tu aider Laeticia à ...
	} % titre de l'activité



\begin{document}
%----------------------------------------------------------------------------------------
%	TITLE PAGE
%----------------------------------------------------------------------------------------
\date{}
\title{\titreActivite}
\maketitle % Print the title page

%----------------------------------------------------------------------------------------
%	QUESTION 1
%----------------------------------------------------------------------------------------

\underline{\textbf{Objectif}} :  \vspace{2pt}
\objectif

\vspace{4pt}

\underline{\textbf{Contexte}} :  \textit{\contexte}

\begin{center}
	\includegraphics[width=0.55\columnwidth]{activité.jpg} % Example image
\end{center}

\textbf{\resumeContexte}
\vspace{-12pt}
%% ----------------------------------

\assignmentSection{Votre mission travail}

\underline{Matériel} : un générateur (sur 6V) - des fils de connexion - deux lampes (L 1 et L 2) - un interrupteur - une
pile - un morceau de laine de fer

%----------------------------------------------------------------------------------------
%	QUESTION 1
%----------------------------------------------------------------------------------------


\begin{question}
	\questiontext{
		Comprendre les effets et le danger du court-circuit.
	}
	\subquestion{
		Protocole 1 :	\textbf{Schématiser} un circuit en 
		série comprenant une pile, deux lampes et un interrupteur fermé :
	}
	\vspace{3cm}
	\subquestion{
		Sur le schéma \textbf{noter}, au crayon à papier,
		A et B les deux bornes de la lampe L 1.
	}
	\subquestion{
		\textbf{Réaliser} le circuit puis \textbf{faire vérifier}.
	}
	\subquestion{
		\textbf{Noter} l'état de la lampe.
		\dotfill
	}			
	% \answerbox{1}
	
	\begin{mybox}[Définition]
		Un dipôle est court-circuité lorsque ses deux
		bornes sont directement reliées par un fil de connexion
		ou tout autre matériaux conducteur.
	\end{mybox}

	\subquestion{
		En te servant de la définition ci dessus, \textbf{court-circuiter} la lampe
		L1 en plaçant un fil entre A et B.	
	}
	\subquestion{
		Protocole 2 : \textbf{Ranger} tout le matériel 
		sauf la pile et le morceau de laine de fer 
		puis \textbf{proposer} un moyen simple de court
		circuiter la pile.
	}
	\answerbox{2}

	\subquestion{
		\textbf{Faire valider} la proposition. Puis \textbf{réaliser} le circuit.
	}
%----------------------------------------------------------------------------------------
%	QUESTION 2
%----------------------------------------------------------------------------------------

\begin{question}
	\questiontext{
		Et le corps humain dans tout ça ?
	}
	\end{question}
	
	\subquestion{
		Lire les documents fournis. 
		(informations sur les dangers liés à 
		l'électrocution et l'électrisation à fournir aux élèves).
	}
	\subquestion{
		\textbf{Expliquer} la différence fondamentale entre
	une électrisation et une électrocution ?
	}
	\answerbox{2}
	
	\subquestion{
		\textbf{Expliquer} pourquoi il n'est pas dangereux de toucher
	 les bornes d'une pile avec les doigts ?
	 }
	\answerbox{2}
	
	\subquestion{ Qu'appelle-t-on la tension du secteur ?}
	\answerbox{2}
	
	\subquestion{
		\textbf{Expliquer} pourquoi est-ce qu'il est dangereux 
		de mettre les doigts dans une prise ?}
	\answerbox{2}
	
	\subquestion{\textbf{Expliquer} le rôle d'un cache-prise ?}
	\answerbox{1}
	\subquestion{ Un cache-prise peut-il être en métal ? \textbf{Argumenter}.}
	\answerbox{1}
\end{question}


\end{document}