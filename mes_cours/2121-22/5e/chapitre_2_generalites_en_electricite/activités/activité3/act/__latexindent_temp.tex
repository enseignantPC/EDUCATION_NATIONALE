%%%%%%%%%%%%%%%%%%%%%%%%%%%%%%%%%%%%%%%%%
% Cleese Assignment (For Students)
% LaTeX Template
% Version 2.0 (27/5/2018)
%
% This template originates from:
% http://www.LaTeXTemplates.com
%
% Author:
% Vel (vel@LaTeXTemplates.com)
%
% License:
% CC BY-NC-SA 3.0 (http://creativecommons.org/licenses/by-nc-sa/3.0/)
% 
%%%%%%%%%%%%%%%%%%%%%%%%%%%%%%%%%%%%%%%%%

%----------------------------------------------------------------------------------------
%	PACKAGES AND OTHER DOCUMENT CONFIGURATIONS
%----------------------------------------------------------------------------------------

\documentclass[10pt]{article}
\input{activite.sty} % Include the file specifying the document structure and custom commands
% \usepackage{activite}
%----------------------------------------------------------------------------------------
%	ASSIGNMENT INFORMATION
%----------------------------------------------------------------------------------------

% Required
\newcommand{\assignmentQuestionName}{Question} % The word to be used as a prefix to question numbers; example alternatives: Problem, Exercise
\newcommand{\assignmentClass}{Physique Chimie} % Course/class
\newcommand{\assignmentTitle}{Activité n°3} % Assignment title or name
\newcommand{\assignmentAuthorName}{Chapitre 2} %

%----------------------------------------------------------------------------------------
%	VARIABLES
%----------------------------------------------------------------------------------------

\newcommand{\titreActivite}{Activité 3 : Le court circuit} % titre de l'activité
\newcommand{\objectif}{ 	
	
	\begin{itemize}
		\item Danger du court-circuit et impact sur les autres dipôles.
		\item Trajet du courant électrique dans un circuit avec court-circuit.
		\item Rôle du cache-prise, du fusible et du disjoncteur.
		\item Danger et précautions liés à l'électrisation et l'électrocution.
	\end{itemize}
} % titre de l'activité
\newcommand{\contexte}{
	% Laeticia raconte ...
}
\newcommand{\resumeContexte}{
	% Saura-tu aider Laeticia à ...
	} % titre de l'activité



\begin{document}
%----------------------------------------------------------------------------------------
%	TITLE PAGE
%----------------------------------------------------------------------------------------
\date{}
\title{\titreActivite}
\maketitle % Print the title page

%----------------------------------------------------------------------------------------
%	QUESTION 1
%----------------------------------------------------------------------------------------

\underline{\textbf{Objectif}} :  \vspace{2pt}
\objectif

\vspace{4pt}

\underline{\textbf{Contexte}} :  \textit{\contexte}

\begin{center}
	\includegraphics[width=0.55\columnwidth]{activité.jpg} % Example image
\end{center}

\textbf{\resumeContexte}
\vspace{-12pt}
%% ----------------------------------

\assignmentSection{Votre mission travail}

\underline{Matériel} : un générateur (sur 6V) - des fils de connexion - deux lampes (L 1 et L 2) - un interrupteur - une
pile - un morceau de laine de fer

%----------------------------------------------------------------------------------------
%	QUESTION 1
%----------------------------------------------------------------------------------------


\begin{question}
	\questiontext{
		Comprendre les effets et le danger du court-circuit.
	}
	\subquestion{
		Protocole 1 :	\textbf{Schématiser} un circuit en 
		série comprenant une pile, deux lampes et un interrupteur fermé :
	}
	\vspace{3cm}
	\subquestion{
		Sur le schéma \textbf{noter}, au crayon à papier,
		A et B les deux bornes de la lampe L 1.
	}
	\subquestion{
		\textbf{Réaliser} le circuit puis \textbf{faire vérifier}.
	}
	\subquestion{
		\textbf{Noter} l'état de la lampe.
		\dotfill
	}			
	% \answerbox{1}
	
	\begin{mybox}[Définition]
		Un dipôle est court-circuité lorsque ses deux
		bornes sont directement reliées par un fil de connexion
		ou tout autre matériaux conducteur.
	\end{mybox}

	\subquestion{
		En te servant de la définition ci dessus, \textbf{court-circuiter} la lampe
		L1 en plaçant un fil entre A et B.	
	}
	\subquestion{
		Protocole 2 : \textbf{Ranger} tout le matériel 
		sauf la pile et le morceau de laine de fer 
		puis \textbf{proposer} un moyen simple de court
		circuiter la pile.
	}
	\answerbox{2}

	\subquestion{
		\textbf{Faire valider} la proposition. Puis \textbf{réaliser} le circuit.
	}
%----------------------------------------------------------------------------------------
%	QUESTION 2
%----------------------------------------------------------------------------------------

\begin{question}
	\questiontext{
		Et le corps humain dans tout ça ? \textbf{Répondre} aux
		questions suivantes à l'aide des document en fin d'activité.
	}
	\end{question}
	
	\subquestion{
		(informations sur les dangers liés à 
		l'électrocution et l'électrisation à fournir aux élèves).
	}
	\subquestion{
		\textbf{Expliquer} la différence fondamentale entre
	une électrisation et une électrocution ?
	}
	\answerbox{2}
		
	\subquestion{ Qu'appelle-t-on la tension du secteur ?}
	\answerbox{2}
	
	\subquestion{
		\textbf{Expliquer} pourquoi est-ce qu'il est dangereux 
		de mettre les doigts dans une prise ?}
	\answerbox{2}
	
\end{question}


\textbf{\color{DarkBlue} {Document 1 : L'électrisation }}  

L'électrisation, souvent confondue avec le terme électrocution, correspond au passage d'un courant
électrique dans le corps d'un Homme ou d'un animal pouvant alors provoquer atteinte aux différents
tissus et organes sur son trajet. L'électrisation peut-être le fruit d'un accident ou encore provoquée.
On peut parler de foudroiement ou de fulguration, au sens large, lorsque l'électrisation est
provoquée par un courant de foudre. Quand on parle de la fulguration dans son terme restreint et en
technique cardiologique, on parle de l'électrisation thérapeutique, donc le défibrillateur, en secours
d'urgence.

\vspace{5pt}
\textbf{\color{DarkBlue} {Document 1 : L'électrocution }}  

L'électrocution correspond au fait de causer une secousse mortelle par le passage d'un courant
électrique chez l'Homme ou chez l'animal. Dans le cas défavorable où le corps est traversé pendant
une seconde par un courant alternatif de 75 mA à 50 voire 60 Hz, une fibrillation ventriculaire peut
être causée et létale sauf si une intervention très rapide à lieu sur le blessé.

\vspace{5pt}
\textbf{\color{DarkBlue} {Document 1 : Les cas en France }}  
 
Chaque année en France, on dénombre 200 personnes hospitalisés suite à une électrisation, c'est à
dire à causes de brûlures électrique ce qui fait un total de 3 à 5 cas par millions d'habitants et par
an.
Les électrocutions représentent les cas d'accidents de la vie courante les plus rares en France
puisqu'on ne dénombrait en 2006 que 61 décès par électrocutions sur un total de 18 000 décès par
accidents de la vie courante.
De façon générale, les accidents d'électrocutions représentent environ deux cas sur trois d'accidents
domestiques et de loisirs et touche de façon majoritaire les hommes, surtout les adultes bricoleurs,
mais aussi les jeunes enfants de moins de 5 ans mis à proximité d'installations défectueuses.



\end{document}