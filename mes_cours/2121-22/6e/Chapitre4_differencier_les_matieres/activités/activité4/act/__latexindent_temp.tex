%%%%%%%%%%%%%%%%%%%%%%%%%%%%%%%%%%%%%%%%%
% Cleese Assignment (For Students)
% LaTeX Template
% Version 2.0 (27/5/2018)
%
% This template originates from:
% http://www.LaTeXTemplates.com
%
% Author:
% Vel (vel@LaTeXTemplates.com)
%
% License:
% CC BY-NC-SA 3.0 (http://creativecommons.org/licenses/by-nc-sa/3.0/)
% 
%%%%%%%%%%%%%%%%%%%%%%%%%%%%%%%%%%%%%%%%%

%----------------------------------------------------------------------------------------
%	PACKAGES AND OTHER DOCUMENT CONFIGURATIONS
%----------------------------------------------------------------------------------------

\documentclass[10pt]{article}
\input{activite.sty} % Include the file specifying the document structure and custom commands
% \usepackage{activite}
%----------------------------------------------------------------------------------------
%	ASSIGNMENT INFORMATION
%----------------------------------------------------------------------------------------

% Required
\newcommand{\assignmentQuestionName}{Question} % The word to be used as a prefix to question numbers; example alternatives: Problem, Exercise
\newcommand{\assignmentClass}{Physique Chimie} % Course/class
\newcommand{\assignmentTitle}{Activité n°4} % Assignment title or name
\newcommand{\assignmentAuthorName}{Chapitre 4} %

%----------------------------------------------------------------------------------------
%	VARIABLES
%----------------------------------------------------------------------------------------

\newcommand{\titreActivite}{Activité 1: Densité, mesure de masse, mesure de volume.} % titre de l'activité
\newcommand{\objectif}{ 	
	
	\begin{itemize}
		\item Savoir ce qu'est une densité.
		\item Savoir mesurer la densité d'un liquide.
	\end{itemize}
} % titre de l'activité
\newcommand{\contexte}{
	M. Feng, professeur de français vers le sud d'Aulnay, travaille une pièce de théatre avec ses élèves. 
	Pendant la dernière scène, pour représenter un tsunami, il veut déverser 1000 litres d'eau salée sur les spectateurs et les participants!
	Après avoir eu l'accord de sa chef d'établissement, il est chargé d'organiser l'achat et le transport de l'eau salée.

	Il utilise sa propre voiture qui peut déplacer 500 kilogramme de matière mais il ne sait pas combien pèse 1000 litres d'eau salée.
	Il ne veut pas risquer un accident de voiture...
}
\newcommand{\resumeContexte}{
	Saura-tu aider M. Feng à déterminer combien de kilogrammes pèsent 1000 litres d'eau salée?
	} % titre de l'activité




%----------------------------------------------------------------------------------------

\begin{document}
%----------------------------------------------------------------------------------------
%	TITLE PAGE
%----------------------------------------------------------------------------------------
\date{}
\title{\titreActivite}
\maketitle % Print the title page

%----------------------------------------------------------------------------------------
%	QUESTION 1
%----------------------------------------------------------------------------------------

\underline{\textbf{Objectif}} :  \vspace{2pt}
\objectif

\vspace{4pt}

\underline{\textbf{Contexte}} :  \textit{\contexte}

\begin{center}
	\includegraphics[width=0.55\columnwidth]{example-image} % Example image
\end{center}

\textbf{\resumeContexte}
\vspace{-12pt}
%% ----------------------------------

\assignmentSection{Votre mission travail}

%----------------------------------------------------------------------------------------
%	QUESTION 1
%----------------------------------------------------------------------------------------

\begin{question}
	\questiontext{
		Question
	}
	\end{question}
	\answerbox{2}

\end{document}
