%%%%%%%%%%%%%%%%%%%%%%%%%%%%%%%%%%%%%%%%%
% Cleese Assignment (For Students)
% LaTeX Template
% Version 2.0 (27/5/2018)
%
% This template originates from:
% http://www.LaTeXTemplates.com
%
% Author:
% Vel (vel@LaTeXTemplates.com)
%
% License:
% CC BY-NC-SA 3.0 (http://creativecommons.org/licenses/by-nc-sa/3.0/)
% 
%%%%%%%%%%%%%%%%%%%%%%%%%%%%%%%%%%%%%%%%%

%----------------------------------------------------------------------------------------
%	PACKAGES AND OTHER DOCUMENT CONFIGURATIONS
%----------------------------------------------------------------------------------------

\documentclass[10pt]{article}
\input{activite.sty} % Include the file specifying the document structure and custom commands
% \usepackage{activite}
%----------------------------------------------------------------------------------------
%	ASSIGNMENT INFORMATION
%----------------------------------------------------------------------------------------

% Required
\newcommand{\assignmentQuestionName}{Question} % The word to be used as a prefix to question numbers; example alternatives: Problem, Exercise
\newcommand{\assignmentClass}{Physique Chimie} % Course/class
\newcommand{\assignmentTitle}{Activité n°1} % Assignment title or name
\newcommand{\assignmentAuthorName}{Chapitre 3} %

%----------------------------------------------------------------------------------------
%	VARIABLES
%----------------------------------------------------------------------------------------

\newcommand{\titreActivite}{Activité 1: La masse} % titre de l'activité
\newcommand{\objectif}{ 	
	
	\begin{itemize}
		\item Comprendre la notion de masse
		\item Savoir mesurer une masse.
	\end{itemize}
} % titre de l'activité
\newcommand{\contexte}{
	Shaynice, bricoleuse dans l’âme, a décidé d’installer deux nouvelles étagères
	dans sa chambre. Elle veut gagner de l’espace et souhaite poser sur ces
	étagères ses classeurs avec ses cours des années précédentes ainsi que
	diverses affaires (des livres, des manuels et son globe terrestre).
	Sur la notice d’installation de ces étagères, il est marqué :
	« La masse maximale autorisée sur chacune des étagères est de 12 kg »
	Shaynice se demande alors si elle va pouvoir déposer tout ce qu’elle aurait voulu, 
}
\newcommand{\resumeContexte}{
	Saura-tu aider Shaynice à installer ses étagères en toute sécurité?
	}




%----------------------------------------------------------------------------------------

\begin{document}
%----------------------------------------------------------------------------------------
%	TITLE PAGE
%----------------------------------------------------------------------------------------
\date{}
\title{\titreActivite}
\maketitle % Print the title page

%----------------------------------------------------------------------------------------
%	QUESTION 1
%----------------------------------------------------------------------------------------

\underline{\textbf{Objectif}} :  \vspace{2pt}
\objectif

\vspace{4pt}

\underline{\textbf{Contexte}} :  \textit{\contexte}

\begin{center}
	\includegraphics[width=0.55\columnwidth]{assets/img-005.png} % Example image
\end{center}

\textbf{\resumeContexte}
\vspace{-12pt}
%% ----------------------------------

\assignmentSection{Votre mission travail}

%----------------------------------------------------------------------------------------
%	QUESTION 1
%----------------------------------------------------------------------------------------

\begin{question}
	\questiontext{
		Nous allons nous poser la même question avec nos trousses.
	}
	\begin{itemize}
		\item Par binôme, choisir une trousse, la vider et y insérer trois objets (stylo, gomme, paire de ciseaux,...)
		\item Classez les objets selon leur masse : de la plus petite à la plus grande, et cela
		sans aucune mesure.
	\end{itemize}
\end{question}


\begin{question}
	\questiontext{
		Pour vérifier mon classement, je vais utiliser une ........................... , et noter les mesures dans le tableau suivant :
		}
\end{question}


\begin{center}
	\begin{tabular}{ |c|c|  }
		\hline
		\makebox(0,15){}\makebox(200,10){Nom de l’objet}  &	 \makebox(0,15){}\makebox(200,10){Masse (en g )} \\\hline
		\makebox(0,15){Trousse} &  \\\hline
		\makebox(0,15){} &  \\\hline
		\makebox(0,15){} &  \\\hline
		\makebox(0,15){} &  \\\hline
	\end{tabular}
\end{center}


	\begin{question}
		\questiontext{
			Comment prévoir la masse de la trousse avec ses trois objets avant de faire la mesure ? \textbf{Justifier}.
			}
	\end{question}
	\answerbox{2}



	\begin{question}
		\questiontext{
			Pour passer des grammes (mesuré par la balance) aux kilogrammes (sur la notice de l'étagère) 
			il faut utiliser un \textbf{tableau de conversion} comme ci dessous.

			Entraîne toi sur les examples suivants :
			\begin{itemize}
				\item La masse d’un cahier : 0,11 kg = ................ g
				\item La masse d’une baguette de pain : 0,25 kg = ................ g
				\item La masse d’un homme : 70 000 g = .................. kg
				\item La masse d’un oreille : 14 hg = .............. kg = .................. g
				\item La masse d’une canette de soda : 384 g = .......... kg = ................. dag
			\end{itemize}
			}
	\end{question}

	\begin{center}
		\begin{tabular}{ |c|c|c|c|c|c|c|  }
			\hline
			\multicolumn{7}{|c|}{Tableau de conversion} \\\hline
			\makebox(0,15){}Kilogramme & Hectogramme & Décagramme & Gramme & Décigramme & Centigramme & Milligramme \\\hline% \makebox(0,15){}
			\makebox(0,15){} &  &  &  &  &  &  \\\hline% 
			\makebox(0,15){} &  &  &  &  &  &  \\\hline% 
			\makebox(0,15){} &  &  &  &  &  &  \\\hline% 
			\makebox(0,15){} &  &  &  &  &  &  \\\hline% 
			\makebox(0,15){} &  &  &  &  &  &  \\\hline% 
			\makebox(0,15){} &  &  &  &  &  &  \\\hline% 
			\makebox(0,15){} &  &  &  &  &  &  \\\hline% 
			\makebox(0,15){} &  &  &  &  &  &  \\\hline% 
		\end{tabular}
	\end{center}


	\begin{question}
		\questiontext{
			Shaynice a pesé 
			\begin{itemize}
				\item chacun de ses quatre classeurs (masse d’un classeur = 1,2 kg).
				\item Ses manuels (masse du premier manuel = 1 560 g, masse du deuxième manuel = 1,4 kg, masse du troisième manuel = 1,6 kg).
				\item Son globe terrestre (masse du globe terrestre = 350 g).
				\item Ses dix mangas (masse d’un manga = 213 g).
			\end{itemize}
		Pourra t-elle poser ses objets sur l'étagère?
		}
	\end{question}
	\answerbox{6}

	

\end{document}
