%%%%%%%%%%%%%%%%%%%%%%%%%%%%%%%%%%%%%%%%%
% Cleese Assignment (For Students)
% LaTeX Template
% Version 2.0 (27/5/2018)
%
% This template originates from:
% http://www.LaTeXTemplates.com
%
% Author:
% Vel (vel@LaTeXTemplates.com)
%
% License:
% CC BY-NC-SA 3.0 (http://creativecommons.org/licenses/by-nc-sa/3.0/)
% 
%%%%%%%%%%%%%%%%%%%%%%%%%%%%%%%%%%%%%%%%%

%----------------------------------------------------------------------------------------
%	PACKAGES AND OTHER DOCUMENT CONFIGURATIONS
%----------------------------------------------------------------------------------------



\documentclass[10pt]{article}

%%%%%%%%%%%%%%%%%%%%%%%%%%%%%%%%%%%%%%%%%
% Cleese Assignment (For Students)
% LaTeX Template
% Version 2.0 (27/5/2018)
%
% This template originates from:
% http://www.LaTeXTemplates.com
%
% Author:
% Vel (vel@LaTeXTemplates.com)
%
% License:
% CC BY-NC-SA 3.0 (http://creativecommons.org/licenses/by-nc-sa/3.0/)
% 
%%%%%%%%%%%%%%%%%%%%%%%%%%%%%%%%%%%%%%%%%

%----------------------------------------------------------------------------------------
%	PACKAGES AND OTHER DOCUMENT CONFIGURATIONS
%----------------------------------------------------------------------------------------



\documentclass[24pt]{article}

%%%%%%%%%%%%%%%%%%%%%%%%%%%%%%%%%%%%%%%%%
% Cleese Assignment (For Students)
% LaTeX Template
% Version 2.0 (27/5/2018)
%
% This template originates from:
% http://www.LaTeXTemplates.com
%
% Author:
% Vel (vel@LaTeXTemplates.com)
%
% License:
% CC BY-NC-SA 3.0 (http://creativecommons.org/licenses/by-nc-sa/3.0/)
% 
%%%%%%%%%%%%%%%%%%%%%%%%%%%%%%%%%%%%%%%%%

%----------------------------------------------------------------------------------------
%	PACKAGES AND OTHER DOCUMENT CONFIGURATIONS
%----------------------------------------------------------------------------------------



\documentclass[24pt]{article}

%%%%%%%%%%%%%%%%%%%%%%%%%%%%%%%%%%%%%%%%%
% Cleese Assignment (For Students)
% LaTeX Template
% Version 2.0 (27/5/2018)
%
% This template originates from:
% http://www.LaTeXTemplates.com
%
% Author:
% Vel (vel@LaTeXTemplates.com)
%
% License:
% CC BY-NC-SA 3.0 (http://creativecommons.org/licenses/by-nc-sa/3.0/)
% 
%%%%%%%%%%%%%%%%%%%%%%%%%%%%%%%%%%%%%%%%%

%----------------------------------------------------------------------------------------
%	PACKAGES AND OTHER DOCUMENT CONFIGURATIONS
%----------------------------------------------------------------------------------------



\documentclass[24pt]{article}

\input{cours.sty} % Include the file specifying the document structure and custom commands

    

%----------------------------------------------------------------------------------------
%	VARIABLES
%----------------------------------------------------------------------------------------

\newcommand{\titre}{Chapitre 2: La matière} % titre de l'activité

%----------------------------------------------------------------------------------------

\begin{document}

\thispagestyle{fancy}
\cfoot{}

\begin{titlebox}{\titre}
    \setlength\parindent{4pt} % Removes all indentation from paragraphs
    \setlength\parskip{5pt} 

    \tcbsubtitle{Activité 1}
    La matière qui nous entoure possède différentes \textbf{\color{DarkRed} {propriétés}} 
    qui la décrivent et aide à différencier les différents types de matière qui existe.

    La matière peut être : 
    \begin{itemize}
        \item \textbf{\color{DarkRed} {Vivante (ou organique)}} : 
        C'est la matière dont est fait tous le monde du vivant
        (animaux, plantes, etc.)
        \item \textbf{\color{DarkRed} {Inerte }} :
        C'est la matière qui n'est pas organique.
        \item \textbf{\color{DarkRed} {Artificielle}} : 
        C'est la matière qui est crée par les humains.
        \item \textbf{\color{DarkRed} {Naturelle }} :
        C'est la matière qui n'est pas crée artificiellement par les humains.
        \item \textbf{\color{DarkRed} {Conductrice}} :
        C'est la matière qui n'empêche pas l'électricité de passer.
        \item \textbf{\color{DarkRed} {Isolante }} :
        C'est la matière qui n'est pas conductrice.
    \end{itemize}
    
    \vspace{30pt}
    \tcbsubtitle{Activité 2}

    Dans l'univers, la matière se trouve dans trois états différents :
    \begin{itemize}
        \item L'état \textbf{\color{DarkRed} {Solide}}: ne change pas de forme, peut être saisi.
        \item L'état \textbf{\color{DarkRed} {Liquide}}: la forme s'adapte au récipient, ne \textbf{peut pas} être saisi.
        \item L'état \textbf{\color{DarkRed} {Gazeux}}: prends tout l'espace disponible, ne \textbf{peut pas} être saisi.
    \end{itemize}
    
    Certains gaz sont invisibles comme la vapeur d'eau par exemple.

    \vspace{30pt}
    \tcbsubtitle{Activité 3}
    Les atomes sont les plus petits "lego" de l'univers (la suite en 4$^{e}$).
    Ils sont crées au centre des étoiles, leur combinaison produit les différents types de matière
    présent dans l'univers.

\end{titlebox}


\end{document}
 % Include the file specifying the document structure and custom commands

    

%----------------------------------------------------------------------------------------
%	VARIABLES
%----------------------------------------------------------------------------------------

\newcommand{\titre}{Chapitre 2: La matière} % titre de l'activité

%----------------------------------------------------------------------------------------

\begin{document}

\thispagestyle{fancy}
\cfoot{}

\begin{titlebox}{\titre}
    \setlength\parindent{4pt} % Removes all indentation from paragraphs
    \setlength\parskip{5pt} 

    \tcbsubtitle{Activité 1}
    La matière qui nous entoure possède différentes \textbf{\color{DarkRed} {propriétés}} 
    qui la décrivent et aide à différencier les différents types de matière qui existe.

    La matière peut être : 
    \begin{itemize}
        \item \textbf{\color{DarkRed} {Vivante (ou organique)}} : 
        C'est la matière dont est fait tous le monde du vivant
        (animaux, plantes, etc.)
        \item \textbf{\color{DarkRed} {Inerte }} :
        C'est la matière qui n'est pas organique.
        \item \textbf{\color{DarkRed} {Artificielle}} : 
        C'est la matière qui est crée par les humains.
        \item \textbf{\color{DarkRed} {Naturelle }} :
        C'est la matière qui n'est pas crée artificiellement par les humains.
        \item \textbf{\color{DarkRed} {Conductrice}} :
        C'est la matière qui n'empêche pas l'électricité de passer.
        \item \textbf{\color{DarkRed} {Isolante }} :
        C'est la matière qui n'est pas conductrice.
    \end{itemize}
    
    \vspace{30pt}
    \tcbsubtitle{Activité 2}

    Dans l'univers, la matière se trouve dans trois états différents :
    \begin{itemize}
        \item L'état \textbf{\color{DarkRed} {Solide}}: ne change pas de forme, peut être saisi.
        \item L'état \textbf{\color{DarkRed} {Liquide}}: la forme s'adapte au récipient, ne \textbf{peut pas} être saisi.
        \item L'état \textbf{\color{DarkRed} {Gazeux}}: prends tout l'espace disponible, ne \textbf{peut pas} être saisi.
    \end{itemize}
    
    Certains gaz sont invisibles comme la vapeur d'eau par exemple.

    \vspace{30pt}
    \tcbsubtitle{Activité 3}
    Les atomes sont les plus petits "lego" de l'univers (la suite en 4$^{e}$).
    Ils sont crées au centre des étoiles, leur combinaison produit les différents types de matière
    présent dans l'univers.

\end{titlebox}


\end{document}
 % Include the file specifying the document structure and custom commands

    

%----------------------------------------------------------------------------------------
%	VARIABLES
%----------------------------------------------------------------------------------------

\newcommand{\titre}{Chapitre 2: La matière} % titre de l'activité

%----------------------------------------------------------------------------------------

\begin{document}

\thispagestyle{fancy}
\cfoot{}

\begin{titlebox}{\titre}
    \setlength\parindent{4pt} % Removes all indentation from paragraphs
    \setlength\parskip{5pt} 

    \tcbsubtitle{Activité 1}
    La matière qui nous entoure possède différentes \textbf{\color{DarkRed} {propriétés}} 
    qui la décrivent et aide à différencier les différents types de matière qui existe.

    La matière peut être : 
    \begin{itemize}
        \item \textbf{\color{DarkRed} {Vivante (ou organique)}} : 
        C'est la matière dont est fait tous le monde du vivant
        (animaux, plantes, etc.)
        \item \textbf{\color{DarkRed} {Inerte }} :
        C'est la matière qui n'est pas organique.
        \item \textbf{\color{DarkRed} {Artificielle}} : 
        C'est la matière qui est crée par les humains.
        \item \textbf{\color{DarkRed} {Naturelle }} :
        C'est la matière qui n'est pas crée artificiellement par les humains.
        \item \textbf{\color{DarkRed} {Conductrice}} :
        C'est la matière qui n'empêche pas l'électricité de passer.
        \item \textbf{\color{DarkRed} {Isolante }} :
        C'est la matière qui n'est pas conductrice.
    \end{itemize}
    
    \vspace{30pt}
    \tcbsubtitle{Activité 2}

    Dans l'univers, la matière se trouve dans trois états différents :
    \begin{itemize}
        \item L'état \textbf{\color{DarkRed} {Solide}}: ne change pas de forme, peut être saisi.
        \item L'état \textbf{\color{DarkRed} {Liquide}}: la forme s'adapte au récipient, ne \textbf{peut pas} être saisi.
        \item L'état \textbf{\color{DarkRed} {Gazeux}}: prends tout l'espace disponible, ne \textbf{peut pas} être saisi.
    \end{itemize}
    
    Certains gaz sont invisibles comme la vapeur d'eau par exemple.

    \vspace{30pt}
    \tcbsubtitle{Activité 3}
    Les atomes sont les plus petits "lego" de l'univers (la suite en 4$^{e}$).
    Ils sont crées au centre des étoiles, leur combinaison produit les différents types de matière
    présent dans l'univers.

\end{titlebox}


\end{document}
 % Include the file specifying the document structure and custom commands
\usepackage{ifthen}


\geometry{
	paper=a4paper, % Change to letterpaper for US letter
	top=   2cm, % Top margin
	bottom=2cm, % Bottom margin
	left = 2cm, % Left margin
	right= 2cm, % Right margin
	headheight=14pt, % Header height
	footskip=1.4cm, % Space from the bottom margin to the baseline of the footer
	headsep=1.2cm, % Space from the top margin to the baseline of the header
	%showframe, % Uncomment to show how the type block is set on the page
}
    

%----------------------------------------------------------------------------------------
%	VARIABLES
%----------------------------------------------------------------------------------------
\newcommand{\titre}{Activité 1} % titre de l'activité
\newcommand{\sep}{-4pt} % titre de l'activité
\newcommand{\DoItNTimes}{5} % titre de l'activité

%----------------------------------------------------------------------------------------

\begin{document}

\thispagestyle{fancy}
\cfoot{}

% \multido{}{3}{\noindent\makebox[\linewidth]{\dotfill}\\}

\newcounter{int}
\setcounter{int}{1}
\loop

\begin{mybox}{\textbf{\titre}}    
    \setlength\parindent{4pt} % Removes all indentation from paragraphs
    \setlength\parskip{5pt} 

	La matière qui nous entoure possède différentes \textbf{\color{DarkRed} {propriétés}} 
    qui la décrivent et aide à différencier les différents types de matière qui existe.

    La matière peut être : 
    \begin{itemize}
        \item \textbf{\color{DarkRed} {Vivante (ou organique)}} : 
        C'est la matière dont est fait tous le monde du vivant
        (animaux, plantes, etc.)
        \item \textbf{\color{DarkRed} {Inerte }} :
        C'est la matière qui n'est pas organique.
        \item \textbf{\color{DarkRed} {Artificielle}} : 
        C'est la matière qui est crée par les humains.
        \item \textbf{\color{DarkRed} {Naturelle }} :
        C'est la matière qui n'est pas crée artificiellement par les humains.
        \item \textbf{\color{DarkRed} {Conductrice}} :
        C'est la matière qui n'empêche pas l'électricité de passer.
        \item \textbf{\color{DarkRed} {Isolante }} :
        C'est la matière qui n'est pas conductrice.
    \end{itemize}


\end{mybox}

\vspace{\sep}

\addtocounter{int}{1}
\ifnum\value{int}<\DoItNTimes\repeat


\end{document}
