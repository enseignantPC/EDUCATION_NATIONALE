%%%%%%%%%%%%%%%%%%%%%%%%%%%%%%%%%%%%%%%%%
% Cleese Assignment (For Students)
% LaTeX Template
% Version 2.0 (27/5/2018)
%
% This template originates from:
% http://www.LaTeXTemplates.com
%
% Author:
% Vel (vel@LaTeXTemplates.com)
%
% License:
% CC BY-NC-SA 3.0 (http://creativecommons.org/licenses/by-nc-sa/3.0/)
% 
%%%%%%%%%%%%%%%%%%%%%%%%%%%%%%%%%%%%%%%%%

%----------------------------------------------------------------------------------------
%	PACKAGES AND OTHER DOCUMENT CONFIGURATIONS
%----------------------------------------------------------------------------------------

\documentclass[10pt]{article}
\input{activite.sty} % Include the file specifying the document structure and custom commands
% \usepackage{activite}
%----------------------------------------------------------------------------------------
%	ASSIGNMENT INFORMATION
%----------------------------------------------------------------------------------------

% Required
\newcommand{\assignmentQuestionName}{Question} % The word to be used as a prefix to question numbers; example alternatives: Problem, Exercise
\newcommand{\assignmentClass}{Physique Chimie} % Course/class
\newcommand{\assignmentTitle}{Activité n°3} % Assignment title or name
\newcommand{\assignmentAuthorName}{Chapitre 2} %

%----------------------------------------------------------------------------------------
%	VARIABLES
%----------------------------------------------------------------------------------------

\newcommand{\titreActivite}{\huge Activité 3: La matière dans l’univers.} % titre de l'activité
\newcommand{\objectif}{ 	
	\begin{itemize}
		\item Savoir d'où provient la matière dans l'univers.
	\end{itemize}
} % titre de l'activité
\newcommand{\contexte}{
	Dans son livre "Poussières d'étoiles" Hubert Reeves a écrit "Nous sommes tous fait de poussières d'étoiles."
	Fatoumata et Marina se sont pas d'accord sur le sens de la phrase. 
	Fatoumata pense que c'est juste une belle phrase mais que ça ne veux pas dire grand chose. 
	Marina pense que la phrase est vraie (même si elle a du mal à comprendre ce que ça veux dire).
}
\newcommand{\resumeContexte}{Sauras-tu aider à départager les deux amies?}


%----------------------------------------------------------------------------------------

\begin{document}
%----------------------------------------------------------------------------------------
%	TITLE PAGE
%----------------------------------------------------------------------------------------
\date{}
\title{\titreActivite}
\maketitle % Print the title page

%----------------------------------------------------------------------------------------
%	QUESTION 1
%----------------------------------------------------------------------------------------

\underline{\textbf{Objectif}} :  \vspace{2pt}
\objectif

\vspace{4pt}



\begin{minipage}[c]{0.45\textwidth}
	\centering \underline{\textbf{Contexte}} :  \textit{\contexte}
\end{minipage}
\hspace{ 0pt}
\begin{minipage}[c]{0.45\textwidth}
	\centering \begin{center}
		\includegraphics[width=0.9\columnwidth]{act.jpeg} % Example image
		\vspace{10pt}
		\textbf{\resumeContexte}
	\end{center}
\end{minipage}


\vspace{-12pt}
%% ----------------------------------

\assignmentSection{Votre mission travail}

{\large À l'aide des documents suivants, répondre aux questions:}

%----------------------------------------------------------------------------------------
%	QUESTION 
%----------------------------------------------------------------------------------------

\begin{question}
	\questiontext{
		Quel est le rapport entre les atomes et la matière qui nous entoure?
	\textbf{Visionner} la vidéo en classe puis \textbf{Répondre.} 
\url{https://www.youtube.com/watch?v=Z6aQC8xxl_A}
	}
	\end{question}

\answerbox{2}

%----------------------------------------------------------------------------------------
%	QUESTION 
%----------------------------------------------------------------------------------------

\begin{question}
	\questiontext{
		Quels étaient initialement les seuls atomes présents dans l'univers?
	}
	\end{question}

	\answerbox{1}
%----------------------------------------------------------------------------------------
%	QUESTION 
%----------------------------------------------------------------------------------------

\begin{question}
	\questiontext{
		Comment se sont formés tous les autres atomes?
	}
	\end{question}
	\answerbox{1}
%----------------------------------------------------------------------------------------
%	QUESTION 
%----------------------------------------------------------------------------------------

\begin{question}
	\questiontext{
		Conclure sur la question du contexte, sommes nous vraiment tous fait de poussières d'étoiles? \textbf{Justifier la réponse.}
	}
	\end{question}
%	\answerbox{}
\answerbox{2}
\begin{minipage}[t]{0.45\textwidth}
	{\centering \textbf{\color{DarkBlue} {Document 1 : Le commencement de l'univers}}}
	
	L'univers apparaît après un évènement appelé le Big Bang (Grand Boum).
	Les premiers atomes sont formés dans les minutes qui suivent, 
	il s'agit exclusivement d'atome d'hydrogène (un gaz abondant dans l'univers encore aujourd'hui)
	et d'atome d'hélium (le gaz utilisé pour gonfler les ballons!).
\end{minipage}
\hspace{20pt}
\begin{minipage}[t]{0.45\textwidth}
	{\centering \textbf{\color{DarkBlue} {Document 2 : La vie des étoiles}}}

	Au début de sa vie, une étoile (comme le soleil par exemple) est une énorme boule
	de gaz. Le gaz s'attire lui même tant et si bien que le c\oe ur de l'étoile est soumis à
	une énorme pression. Sous cette pression on dit que les atomes d'hydrogène vont commencer à fusionner 
	les uns avec les autres pour former d'autres atomes qui fusionneront eux aussi pour donner 
	la majorité des atomes que l'on connaît! Cette matière s'est ensuite répartie dans l'univers
	ce qui a permis de former la terre et toute la matière qu'on trouve dessus.
\end{minipage}
\end{document}
