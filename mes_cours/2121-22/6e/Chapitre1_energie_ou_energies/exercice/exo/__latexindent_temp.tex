%%%%%%%%%%%%%%%%%%%%%%%%%%%%%%%%%%%%%%%%%
% Cleese Assignment (For Students)
% LaTeX Template
% Version 2.0 (27/5/2018)
%
% This template originates from:
% http://www.LaTeXTemplates.com
%
% Author:
% Vel (vel@LaTeXTemplates.com)
%
% License:
% CC BY-NC-SA 3.0 (http://creativecommons.org/licenses/by-nc-sa/3.0/)
% 
%%%%%%%%%%%%%%%%%%%%%%%%%%%%%%%%%%%%%%%%%

%----------------------------------------------------------------------------------------
%	PACKAGES AND OTHER DOCUMENT CONFIGURATIONS
%----------------------------------------------------------------------------------------

\documentclass[10pt]{article}
%%%%%%%%%%%%%%%%%%%%%%%%%%%%%%%%%%%%%%%%%
% Cleese Assignment
% Structure Specification File
% Version 1.0 (27/5/2018)
%
% This template originates from:
% http://www.LaTeXTemplates.com
%
% Author:
% Vel (vel@LaTeXTemplates.com)
%
% License:
% CC BY-NC-SA 3.0 (http://creativecommons.org/licenses/by-nc-sa/3.0/)
% 
%%%%%%%%%%%%%%%%%%%%%%%%%%%%%%%%%%%%%%%%%

%----------------------------------------------------------------------------------------
%	PACKAGES AND OTHER DOCUMENT CONFIGURATIONS
%----------------------------------------------------------------------------------------

\usepackage{lastpage} % Required to determine the last page number for the footer

\usepackage{graphicx} % Required to insert images

\setlength\parindent{0pt} % Removes all indentation from paragraphs

\usepackage[svgnames]{xcolor} % Enabling colors by their 'svgnames'
\usepackage[most]{tcolorbox} % Required for boxes that split across pages

\usepackage{booktabs} % Required for better horizontal rules in tables

\usepackage{listings} % Required for insertion of code

\usepackage{etoolbox} % Required for if statements

\usepackage{multido} % required for dotted lines

\usepackage[french]{babel} % English language hyphenation

\usepackage{svg}

%----------------------------------------------------------------------------------------
%	MARGINS
%----------------------------------------------------------------------------------------

\usepackage{geometry} % Required for adjusting page dimensions and margins

\geometry{
	paper=a4paper, % Change to letterpaper for US letter
	top=3cm, % Top margin
	bottom=3cm, % Bottom margin
	left=2.5cm, % Left margin
	right=2.5cm, % Right margin
	headheight=14pt, % Header height
	footskip=1.4cm, % Space from the bottom margin to the baseline of the footer
	headsep=1.2cm, % Space from the top margin to the baseline of the header
	%showframe, % Uncomment to show how the type block is set on the page
}

%----------------------------------------------------------------------------------------
%	FONT
%----------------------------------------------------------------------------------------
\usepackage{fontspec}
\usepackage{unicode-math}
\usepackage[utf8]{inputenc} % Required for inputting international characters
\usepackage[T1]{fontenc} % Output font encoding for international characters

\setmainfont[Ligatures=TeX]{Caladea}
%----------------------------------------------------------------------------------------
%	HEADERS AND FOOTERS
%----------------------------------------------------------------------------------------

\usepackage{fancyhdr} % Required for customising headers and footers

\pagestyle{fancy} % Enable custom headers and footers

% \lhead{\small\assignmentClass\ifdef{\assignmentClassInstructor}{\ (\assignmentClassInstructor):}{}\ \assignmentTitle} % Left header; output the instructor in brackets if one was set
% \chead{} % Centre header
% \rhead{\small\ifdef{\assignmentAuthorName}{\assignmentAuthorName}{\ifdef{\assignmentDueDate}{Due\ \assignmentDueDate}{}}} % Right header; output the author name if one was set, otherwise the due date if that was set

% \lfoot{} % Left footer
% \cfoot{\small Page\ \thepage\ sur\ \pageref{LastPage}} % Centre footer
% \rfoot{} % Right footer

% \renewcommand\headrulewidth{0.5pt} % Thickness of the header rule


% \fancypagestyle{plain}{
% 	\lhead{} % Left header; output the instructor in brackets if one was set
% 	\chead{} % Centre header
% 	\rhead{}

%   \renewcommand{\headrulewidth}{0pt}%
%   \fancyfoot[C]{\small Page\ \thepage\ sur\ \pageref{LastPage}}%
% }

%----------------------------------------------------------------------------------------
%	MODIFY SECTION STYLES
%----------------------------------------------------------------------------------------

% \usepackage{titlesec} % Required for modifying sections

%------------------------------------------------
% Section

% \titleformat
% {\section} % Section type being modified
% [block] % Shape type, can be: hang, block, display, runin, leftmargin, rightmargin, drop, wrap, frame
% {\Large\bfseries} % Format of the whole section
% {\assignmentQuestionName~\thesection} % Format of the section label
% {6pt} % Space between the title and label
% {} % Code before the label

% \titlespacing{\section}{0pt}{0.5\baselineskip}{0.5\baselineskip} % Spacing around section titles, the order is: left, before and after

%------------------------------------------------
% Subsection

% \titleformat
% {\subsection} % Section type being modified
% [block] % Shape type, can be: hang, block, display, runin, leftmargin, rightmargin, drop, wrap, frame
% {\itshape} % Format of the whole section
% {(\alph{subsection})} % Format of the section label
% {4pt} % Space between the title and label
% {} % Code before the label

% \titlespacing{\subsection}{0pt}{0.5\baselineskip}{0.5\baselineskip} % Spacing around section titles, the order is: left, before and after

% \renewcommand\thesubsection{(\alph{subsection})}

%----------------------------------------------------------------------------------------
%	CUSTOM QUESTION COMMANDS/ENVIRONMENTS
%----------------------------------------------------------------------------------------


%	TITLE SECTION
%----------------------------------------------------------------------------------------

% \newcommand{\authorstyle}[1]{{\large\usefont{OT1}{phv}{b}{n}#1}} % Authors style (Helvetica)

% \newcommand{\institution}[1]{{\footnotesize\usefont{OT1}{phv}{m}{sl}#1}} % Institutions style (Helvetica)

% \usepackage{titling} % Allows custom title configuration

% \newcommand{\HorRule}{\rule{\linewidth}{1pt}} % Defines the gold horizontal rule around the title

% \pretitle{
% 	\centering
% 	\vspace{-120pt} % Move the entire title section up
% 	\HorRule\vspace{10pt} % Horizontal rule before the title
% 	% \textbf
% 	\bfseries
% 	\fontsize{32}{36}
% 	\usefont{T1}{phv}{b}{n}
% 	\selectfont % Helvetica
% 	\color{DarkRed} % Text colour for the title and author(s)
% }

% \posttitle{\par\vskip 15pt} % Whitespace under the title

% \preauthor{} % Anything that will appear before \author is printed

% \postauthor{ % Anything that will appear after \author is printed
% 	% \vspace{10pt} % Space before the rule
% 	\par\HorRule % Horizontal rule after the title
% 	\vspace{-30pt} % Space after the title section
% }

 % Include the file specifying the document structure and custom commands

%----------------------------------------------------------------------------------------
%	ASSIGNMENT INFORMATION
%----------------------------------------------------------------------------------------

% Required
\newcommand{\assignmentQuestionName}{Exercice} % The word to be used as a prefix to question numbers; example alternatives: Problem, Exercise
\newcommand{\assignmentClass}{Physique Chimie} % Course/class
\newcommand{\assignmentTitle}{Exercice} % Assignment title or name
\newcommand{\assignmentAuthorName}{Chapitre 1} %
\newcommand{\titreActivite}{Exercices} % titre de l'activité

%----------------------------------------------------------------------------------------
%	VARIABLES
%----------------------------------------------------------------------------------------


%----------------------------------------------------------------------------------------

\begin{document}
%----------------------------------------------------------------------------------------
%	TITLE PAGE
%----------------------------------------------------------------------------------------
\date{}
% \maketitle % Print the title page

%----------------------------------------------------------------------------------------
%	QUESTION 1
%----------------------------------------------------------------------------------------
% \vspace{-160pt}
\assignmentSection{VOS EXERCICES ENTRAÎNEMENT}


%----------------------------------------------------------------------------------------
%	QUESTION 1
%----------------------------------------------------------------------------------------

\begin{question}
    \questiontext{
        Dans les situations suivantes, \textbf{nommer} les types d'énergie mis en jeu.
    }
    \subquestion{Le radiateur électrique fonctionne à plein régime aujourd'hui. 
    --> \ldots\ldots\ldots\ldots\ldots
        \ldots\ldots\ldots\ldots\ldots
        \ldots\ldots\ldots\ldots\ldots
        \ldots\ldots\ldots\ldots\ldots
    } 
    \subquestion{La lumière du soleil réchauffe ma peau.
    --> \ldots\ldots\ldots\ldots\ldots
        \ldots\ldots\ldots\ldots\ldots
        \ldots\ldots\ldots\ldots\ldots
        \ldots\ldots\ldots\ldots\ldots
        \ldots\ldots\ldots\ldots\ldots
        \ldots\ldots\ldots\ldots\ldots
        \ldots\ldots\ldots
    }
    \subquestion{À l'aide de sa force seule, Hulk peut soulever des voitures
    --> \ldots\ldots\ldots\ldots\ldots
        \ldots\ldots\ldots\ldots\ldots
        \ldots\ldots\ldots\ldots\ldots
        \ldots\ldots\ldots\ldots\ldots
        \ldots\ldots\ldots\ldots
    }
    \subquestion{La pile de ma lampe s'épuise quand elle est allumée.
    --> \ldots\ldots\ldots\ldots\ldots
        \ldots\ldots\ldots\ldots\ldots
        \ldots\ldots\ldots\ldots\ldots
        \ldots\ldots\ldots\ldots\ldots
        \ldots\ldots\ldots\ldots\ldots
    }    
    \subquestion{L'uranium dans le réacteur réchauffe de l'eau.
    --> \ldots\ldots\ldots\ldots\ldots
        \ldots\ldots\ldots\ldots\ldots
        \ldots\ldots\ldots\ldots\ldots
        \ldots\ldots\ldots\ldots\ldots
        \ldots\ldots\ldots\ldots\ldots
        \ldots\ldots\ldots\ldots
    }
\end{question}

%----------------------------------------------------------------------------------------
%	QUESTION 2
%----------------------------------------------------------------------------------------

\begin{question}
    \questiontext{
        Pour les situations suivantes, 
        \textbf{réalise} les diagrammes de conversion d'énergie.
    }
    \subquestion{Une batterie contient de l'énergie chimique et peut
    la convertir en électricité.}
    \subquestion{Un moteur utilise de l'essence pour pousser une voiture.}
    \subquestion{Un panneau solaire.}
    \end{question}
%	\answerbox{}

%----------------------------------------------------------------------------------------
%	QUESTION 3
%----------------------------------------------------------------------------------------

\begin{question}
    \questiontext{
        Pour les situations suivantes, 
        \textbf{réalise} les diagrammes d'énergie.
    }
    \subquestion{L'eau qui coule de la montagne fait tourner 
    l'alternateur du barrage hydroélectrique qui génère alors de l'électricité.
    }
    \subquestion{
        L'uranium enrichi dans la centrale réchauffe de l'eau. L'eau s'évapore. 
        La vapeur d'eau, en s'échappant, entraîne un alternateur qiu génère alors de l'électricité.
    }
    \subquestion{
        Les substances à l'intérieur d'une batterie réagissent pour créer de l'électricité,
        cette électricité passe à travers une ampoule, elle fait chauffer 
        le filament de l'ampoule qui génère alors de la lumière.
    }
    % \subquestion{L'essence brûle dans le moteur d'une voiture, les gazs 
    % génèrés poussent une piece de métal (piston) qui va faire tourner les roues arrières.
    % de la voiture}
    \end{question}
%	\answerbox{}


%----------------------------------------------------------------------------------------
%	QUESTION 4
%----------------------------------------------------------------------------------------

\begin{question}
    \questiontext{
        \textbf{Expliquer} les diagrammes suivants à l'aide d'une phrase.
    }
    \subquestion{}
    \begin{center}
        \tcbincludegraphics[
            title=diagramme d'énergie d'un train à vapeur:,
            colback=white,
            width=12cm,
            ]
            {diagramme1.png}
    \end{center}

    \subquestion{aide : un piston est une pièce mécanique qui quand on la pousse 
    (avec de l'énergie mécanique) fait tourner les roues du 
    train (énergie mécanique).}
    \begin{center}
        \tcbincludegraphics[
            title=diagramme d'énergie d'un train à vapeur:,
            colback=white,
            width=12cm
            ]
            {diagramme2.png}
    \end{center}
    % \begin{center}
    %     \tcbincludegraphics[
    %         title=diagramme d'énergie de la situation:,
    %         colback=white,
    %         ]
    %         {diagramme2.png}
    % \end{center}

    \end{question}
%	\answerbox{}






\end{document}
