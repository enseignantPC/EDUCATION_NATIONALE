%%%%%%%%%%%%%%%%%%%%%%%%%%%%%%%%%%%%%%%%%
% Cleese Assignment (For Students)
% LaTeX Template
% Version 2.0 (27/5/2018)
%
% This template originates from:
% http://www.LaTeXTemplates.com
%
% Author:
% Vel (vel@LaTeXTemplates.com)
%
% License:
% CC BY-NC-SA 3.0 (http://creativecommons.org/licenses/by-nc-sa/3.0/)
% 
%%%%%%%%%%%%%%%%%%%%%%%%%%%%%%%%%%%%%%%%%

%----------------------------------------------------------------------------------------
%	PACKAGES AND OTHER DOCUMENT CONFIGURATIONS
%----------------------------------------------------------------------------------------

\documentclass[10pt]{article}
\input{activite.sty} % Include the file specifying the document structure and custom commands
% \usepackage{activite}
%----------------------------------------------------------------------------------------
%	ASSIGNMENT INFORMATION
%----------------------------------------------------------------------------------------

% Required
\newcommand{\assignmentQuestionName}{Question} % The word to be used as a prefix to question numbers; example alternatives: Problem, Exercise
\newcommand{\assignmentClass}{Physique Chimie} % Course/class
\newcommand{\assignmentTitle}{Activité n°1} % Assignment title or name
\newcommand{\assignmentAuthorName}{Chapitre 3} %

%----------------------------------------------------------------------------------------
%	VARIABLES
%----------------------------------------------------------------------------------------

\newcommand{\titreActivite}{Activité 1: Le Volume} % titre de l'activité
\newcommand{\objectif}{ 	
	
	\begin{itemize}
		\item Notion de volume.
		\item Savoir mesurer un volume.
	\end{itemize}
}
\newcommand{\contexte}{
	Abdel et Sohane ont décidé de préparer des pancakes avec leur mère
	pour le goûter. En ouvrant le livre de recette, nous obtenons la liste des
	ingrédients suivante :
	\begin{itemize}
		\item  2 Œufs.
		\item  15 g de sucre en poudre.
		\item  10 mL d’huile d’olive.
		\item  300 g de farine.
		\item  1 sachet de levure.
		\item  400 mL de lait.
	\end{itemize}
	Ils savent comment obtenir précisément 15 g de sucre ou 300 g de farine mais 
	sont bien embétés pour obtenir 10 mL d’huile d’olive ou 400 mL de lait.
}
\newcommand{\resumeContexte}{
	Saura-tu aider Abdel et Sohane à faire des pancakes?
	}


%----------------------------------------------------------------------------------------

\begin{document}
%----------------------------------------------------------------------------------------
%	TITLE PAGE
%----------------------------------------------------------------------------------------
\date{}
\title{\titreActivite}
\maketitle % Print the title page

%----------------------------------------------------------------------------------------
%	QUESTION 1
%----------------------------------------------------------------------------------------

\begin{minipage}[c]{0.5\textwidth}
	\underline{\textbf{Objectif}} :  \vspace{2pt}
	\objectif

	\vspace{4pt}


    \underline{\textbf{Contexte}} :  \textit{\contexte}
\end{minipage} 
\hspace{10pt}
\begin{minipage}[c]{0.45\textwidth}
	\begin{center}
		\includegraphics[width=0.8\columnwidth]{assets/img_-001.png} % Example image
	\end{center}

	\textbf{\resumeContexte}
\end{minipage}

% \vspace{-12pt}
%% ----------------------------------

\assignmentSection{Votre mission travail}

%----------------------------------------------------------------------------------------
%	QUESTION 1
%----------------------------------------------------------------------------------------

\begin{question}
	\questiontext{
		La mère de nos deux cuisiniers essaye de résoudre leur problème en leur donnant l’ustensile ci-dessous,
		comment celui-ci s’appelle-t-il ? A quoi sert-t-il ?
	}
\end{question}

\hspace{30pt}
\begin{minipage}[c]{0.25\textwidth}
    \includegraphics[width=0.5\textwidth]{assets/img_-002.png}
\end{minipage} 
\hspace{-30pt}
\begin{minipage}[c]{0.75\textwidth}
\answerbox{2}
\end{minipage}


\begin{question}
	\questiontext{
		En chimie, nous utilisons du matériel adapté à la mesure du volume.
		Connais-tu le noms de la verrerie suivante?
	}
\end{question}

\makebox(0,20){}

\begin{minipage}[c]{0.3\textwidth}
	\begin{minipage}[c]{0.45\textwidth}
		\includegraphics[width=\textwidth]{assets/img_-003.png}
	\end{minipage} 
	\begin{minipage}[c]{0.45\textwidth}
		\includegraphics[width=\textwidth]{assets/img_-004.png}
	\end{minipage} 
\end{minipage} \hspace{10pt}
\begin{minipage}[c]{0.65\textwidth}
	\answerbox{3}
\end{minipage}

\newpage

\begin{question}
	\questiontext{
		A partir du tableau de conversion ci-dessous, convertir les volumes donnés dans les unités demandées
\begin{center}
	\begin{itemize}
		\item Une bouteille d’eau : 1L = ............. mL
		\item Une canette de soda de 33 cL = ............. mL = ............. L
		\item Une bassine d’eau : 3500 mL = .............. L
		\item Une baignoire remplie d’eau : 120 L = ........... mL
	\end{itemize}   
\end{center} 
 }
\end{question}

\begin{center}
	\begin{tabular}{ |c|c|c|c|c|c|c|  }
		\hline
		\multicolumn{7}{|c|}{\makebox(0,20){} Tableau de conversion} \\\hline
		\makebox(0,20){} \hspace{10pt} Kilolitre \hspace{10pt} &\hspace{10pt}  Hectolitre \hspace{10pt} &\hspace{10pt}  Décalitre \hspace{10pt} &\hspace{10pt}  litre \hspace{10pt} & \hspace{10pt} Décilitre\hspace{10pt}  & \hspace{10pt} Centilitre \hspace{10pt} & \hspace{10pt} Millilitre\hspace{10pt}  \\\hline% \makebox(0,15){}
		\makebox(0,20){} &  &  &  &  &  &  \\\hline% 
		\makebox(0,20){} &  &  &  &  &  &  \\\hline% 
		\makebox(0,20){} &  &  &  &  &  &  \\\hline% 
		\makebox(0,20){} &  &  &  &  &  &  \\\hline% 
	\end{tabular}
\end{center}

\begin{question}
	\questiontext{
		Indiquer la valeur du volume mesuré pour les éprouvettes a, b, c et d.
	\vspace{10pt}

	\begin{minipage}[c]{0.45\textwidth}
		\begin{center}
			\includegraphics[width=\textwidth]{assets/img_-011.png}
		\end{center}		
	\end{minipage} \hspace{10pt}
	\begin{minipage}[c]{0.5\textwidth}
		Pour lire précisément la graduation concernant la valeur
		de notre volume, il faut :
		Dans une ...................................................., il faut lire la
		graduation qui correspond au bas du ménisque. Le
		ménisque est le creux que forme la surface libre d'un
		liquide.
		\begin{center}
			\includegraphics[width=0.5\textwidth]{assets/img_-013.png}
		\end{center}	
	\end{minipage}
	\begin{center}
		\includegraphics[width=0.5\textwidth]{assets/img_-012.png}
	\end{center}

}
\end{question}
\answerbox{2}

\begin{question}
	\questiontext{
	Mesurer 15 mL d'eau dans une éprouvette puis appeler 
	l'enseignant.e pour vérification.
}
\end{question}

\end{document}
