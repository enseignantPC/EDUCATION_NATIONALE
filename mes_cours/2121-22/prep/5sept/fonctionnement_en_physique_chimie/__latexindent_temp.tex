%%%%%%%%%%%%%%%%%%%%%%%%%%%%%%%%%%%%%%%%%
% Cleese Assignment (For Students)
% LaTeX Template
% Version 2.0 (27/5/2018)
%
% This template originates from:
% http://www.LaTeXTemplates.com
%
% Author:
% Vel (vel@LaTeXTemplates.com)
%
% License:
% CC BY-NC-SA 3.0 (http://creativecommons.org/licenses/by-nc-sa/3.0/)
% 
%%%%%%%%%%%%%%%%%%%%%%%%%%%%%%%%%%%%%%%%%

%----------------------------------------------------------------------------------------
%	PACKAGES AND OTHER DOCUMENT CONFIGURATIONS
%----------------------------------------------------------------------------------------



\documentclass[10pt]{article}

%%%%%%%%%%%%%%%%%%%%%%%%%%%%%%%%%%%%%%%%%
% Cleese Assignment
% Structure Specification File
% Version 1.0 (27/5/2018)
%
% This template originates from:
% http://www.LaTeXTemplates.com
%
% Author:
% Vel (vel@LaTeXTemplates.com)
%
% License:
% CC BY-NC-SA 3.0 (http://creativecommons.org/licenses/by-nc-sa/3.0/)
% 
%%%%%%%%%%%%%%%%%%%%%%%%%%%%%%%%%%%%%%%%%

%----------------------------------------------------------------------------------------
%	PACKAGES AND OTHER DOCUMENT CONFIGURATIONS
%----------------------------------------------------------------------------------------

\usepackage{lastpage} % Required to determine the last page number for the footer

\usepackage{graphicx} % Required to insert images

\setlength\parindent{0pt} % Removes all indentation from paragraphs

\usepackage[svgnames]{xcolor} % Enabling colors by their 'svgnames'
\usepackage[most]{tcolorbox} % Required for boxes that split across pages

\usepackage{booktabs} % Required for better horizontal rules in tables

\usepackage{listings} % Required for insertion of code

\usepackage{etoolbox} % Required for if statements

\usepackage{multido} % required for dotted lines

\usepackage[french]{babel} % English language hyphenation

\usepackage{svg}

%----------------------------------------------------------------------------------------
%	MARGINS
%----------------------------------------------------------------------------------------

\usepackage{geometry} % Required for adjusting page dimensions and margins

\geometry{
	paper=a4paper, % Change to letterpaper for US letter
	top=3cm, % Top margin
	bottom=3cm, % Bottom margin
	left=2.5cm, % Left margin
	right=2.5cm, % Right margin
	headheight=14pt, % Header height
	footskip=1.4cm, % Space from the bottom margin to the baseline of the footer
	headsep=1.2cm, % Space from the top margin to the baseline of the header
	%showframe, % Uncomment to show how the type block is set on the page
}

%----------------------------------------------------------------------------------------
%	FONT
%----------------------------------------------------------------------------------------
\usepackage{fontspec}
\usepackage{unicode-math}
\usepackage[utf8]{inputenc} % Required for inputting international characters
\usepackage[T1]{fontenc} % Output font encoding for international characters

\setmainfont[Ligatures=TeX]{Caladea}
%----------------------------------------------------------------------------------------
%	HEADERS AND FOOTERS
%----------------------------------------------------------------------------------------

\usepackage{fancyhdr} % Required for customising headers and footers

\pagestyle{fancy} % Enable custom headers and footers

% \lhead{\small\assignmentClass\ifdef{\assignmentClassInstructor}{\ (\assignmentClassInstructor):}{}\ \assignmentTitle} % Left header; output the instructor in brackets if one was set
% \chead{} % Centre header
% \rhead{\small\ifdef{\assignmentAuthorName}{\assignmentAuthorName}{\ifdef{\assignmentDueDate}{Due\ \assignmentDueDate}{}}} % Right header; output the author name if one was set, otherwise the due date if that was set

% \lfoot{} % Left footer
% \cfoot{\small Page\ \thepage\ sur\ \pageref{LastPage}} % Centre footer
% \rfoot{} % Right footer

% \renewcommand\headrulewidth{0.5pt} % Thickness of the header rule


% \fancypagestyle{plain}{
% 	\lhead{} % Left header; output the instructor in brackets if one was set
% 	\chead{} % Centre header
% 	\rhead{}

%   \renewcommand{\headrulewidth}{0pt}%
%   \fancyfoot[C]{\small Page\ \thepage\ sur\ \pageref{LastPage}}%
% }

%----------------------------------------------------------------------------------------
%	MODIFY SECTION STYLES
%----------------------------------------------------------------------------------------

% \usepackage{titlesec} % Required for modifying sections

%------------------------------------------------
% Section

% \titleformat
% {\section} % Section type being modified
% [block] % Shape type, can be: hang, block, display, runin, leftmargin, rightmargin, drop, wrap, frame
% {\Large\bfseries} % Format of the whole section
% {\assignmentQuestionName~\thesection} % Format of the section label
% {6pt} % Space between the title and label
% {} % Code before the label

% \titlespacing{\section}{0pt}{0.5\baselineskip}{0.5\baselineskip} % Spacing around section titles, the order is: left, before and after

%------------------------------------------------
% Subsection

% \titleformat
% {\subsection} % Section type being modified
% [block] % Shape type, can be: hang, block, display, runin, leftmargin, rightmargin, drop, wrap, frame
% {\itshape} % Format of the whole section
% {(\alph{subsection})} % Format of the section label
% {4pt} % Space between the title and label
% {} % Code before the label

% \titlespacing{\subsection}{0pt}{0.5\baselineskip}{0.5\baselineskip} % Spacing around section titles, the order is: left, before and after

% \renewcommand\thesubsection{(\alph{subsection})}

%----------------------------------------------------------------------------------------
%	CUSTOM QUESTION COMMANDS/ENVIRONMENTS
%----------------------------------------------------------------------------------------


%	TITLE SECTION
%----------------------------------------------------------------------------------------

% \newcommand{\authorstyle}[1]{{\large\usefont{OT1}{phv}{b}{n}#1}} % Authors style (Helvetica)

% \newcommand{\institution}[1]{{\footnotesize\usefont{OT1}{phv}{m}{sl}#1}} % Institutions style (Helvetica)

% \usepackage{titling} % Allows custom title configuration

% \newcommand{\HorRule}{\rule{\linewidth}{1pt}} % Defines the gold horizontal rule around the title

% \pretitle{
% 	\centering
% 	\vspace{-120pt} % Move the entire title section up
% 	\HorRule\vspace{10pt} % Horizontal rule before the title
% 	% \textbf
% 	\bfseries
% 	\fontsize{32}{36}
% 	\usefont{T1}{phv}{b}{n}
% 	\selectfont % Helvetica
% 	\color{DarkRed} % Text colour for the title and author(s)
% }

% \posttitle{\par\vskip 15pt} % Whitespace under the title

% \preauthor{} % Anything that will appear before \author is printed

% \postauthor{ % Anything that will appear after \author is printed
% 	% \vspace{10pt} % Space before the rule
% 	\par\HorRule % Horizontal rule after the title
% 	\vspace{-30pt} % Space after the title section
% }

 % Include the file specifying the document structure and custom commands
\usepackage{skull}
%----------------------------------------------------------------------------------------
%	VARIABLES
%----------------------------------------------------------------------------------------

\newcommand{\titre}{Fonctionnement en physique chimie} % titre de l'activité
\newcommand{\introduction}{
    La physique chimie est une \emph{science naturelle} dont on se sert pour
    comprendre comment fonctionne le monde.

    Pour que notre classe fonctionne en physique chimie,
    il faut répondre à des questions très importantes:
    \begin{itemize}
        \item Quelles sont les règles à respecter en physique chimie?
        \item Comment réussir en physique chimie?
        \item À quoi ça sert la physique chimie et qu'est-ce que ça peut m'apporter?
    \end{itemize}}

%----------------------------------------------------------------------------------------

\begin{document}
    \thispagestyle{plain}
    \cfoot{}

    \begin{titlebox}{\titre}
        \tcbsubtitle{Introduction:}    
        % for wrapping an icon left or right ----------------------------
        \begin{wrapfigure}[]{L}{3.5cm}
            \vspace{-0.7cm}
            \tcbincludegraphics[width=3cm,colback=white]{logo/electricien_fou.jpeg}
        \end{wrapfigure}%
        % ---------------------------------------------------------------
        \textit{\introduction }
        .% \vspace*{5pt}

    \tcbsubtitle{Quelles sont les règles à respecter en physique chimie?}

    Pour assurer la sécurité physique et émotionnelle chacun en classe (y compris l'enseignant.e) 
    nous avons besoin de règles.

    \textbf{\color{DarkRed} {Règle 1 : Se respecter les uns les autres}}
    \begin{enumerate}
        \item Ne pas se moquer ou se rabaisser d'un.e camarade.
        \item Ne pas être malveillant.e vis à vis d'un.e camarade.
    \end{enumerate}

    \textbf{\color{DarkRed} {Règle 2 : Empêcher le cours d'avoir lieu normalement}}
    \begin{itemize}
        \item Si un élève a un problème parcequ'une des règles n'a pas été respecté,
        essayer de trouver une solution \textbf{par soi même}.    
    \end{itemize}



    \textbf{\color{DarkRed} {Règle 3 : Respecter le cours}}
    \begin{enumerate}
        \item Ne pas rester passif
        \begin{itemize}
            \item Si on a un problème, essayer de le résoudre seul puis demander de l'aide.
            \item suivre sur les affaires d'un.e camarade si on a pas la feuille etc.
        \end{itemize}
        \item Prendre soin de ses affaires
        \begin{itemize}
            \item Avoir ses affaires en classe.
            \item Avoir des affaires propre.
        \end{itemize}
        \item Rattraper le cours \emph{tout seul} si on est absent.e.
        \begin{itemize}
            \item photocopier les documents qui manquent.
            \item rattraper le contenu de la séance.
        \end{itemize}
    \end{enumerate}

    \tcblower

    \begin{wrapfigure}[]{R}{2.8cm}
        \vspace*{-0.5cm}
        \tcbincludegraphics[width=2.5cm,colback=white]{logo/prof_violent.jpeg}
    \end{wrapfigure}

    \textbf{\color{DarkRed} {Conséquences si les règles ne sont pas respectées:}}

    Les conséquences en cas de non respect des règles: 
    \begin{enumerate}
        \item \includegraphics[width=1em]{dessin.pdf} Briser le cœur de son enseignant.e \includegraphics[width=1em]{dessin.pdf}
        \item Suspension d'un privilège
        \textit{La plus légère des punitions}
        \begin{itemize}
            \item L'élève n'a plus le droit de prendre la parole.
            \item L'élève ne peut plus participer à l'expérience faites en classe.
        \end{itemize}
        \item Heure de retenue
        \item $\skull$ Exclusion de cours $\skull$  
    \end{enumerate}

    \tcbsubtitle{
        \begin{center}
            Comment réussir en physique chimie?
        \end{center}
    }

    \begin{itemize}
        \item Participer et poser des questions quand on ne comprend pas
        \item Faire le travail demandé
        \item Poser des questions en rapport avec la physique chimie dans la boite à suggestion
        \item Ces personnages t'accompagneront toute l'année, fais leur confiance:
        
        \tcbincludegraphics[boxrule=0.5pt,equal height group=AT,before=,after=\hfill,height=2cm,width=(\linewidth-2mm)/3,colback=white]{logo/finn.jpg}
        \tcbincludegraphics[boxrule=0.5pt,equal height group=AT,before=,after=\hfill,height=2cm,width=(\linewidth-2mm)/3,colback=white]{logo/one_punch_man_simple.jpg}
        \tcbincludegraphics[boxrule=0.5pt,equal height group=AT,before=,after=\hfill,height=2cm,width=(\linewidth-2mm)/3,colback=white]{logo/hidla_determined.jpg}
        
        \item Ces personnages essaieront de te piéger toute l'année, méfie toi:
        
        \tcbincludegraphics[boxrule=0.5pt,equal height group=AT,before=,after=\hfill,height=2cm,width=(\linewidth-2mm)/3,colback=white]{logo/tomura.jpeg}
        \tcbincludegraphics[boxrule=0.5pt,equal height group=AT,before=,after=\hfill,height=2cm,width=(\linewidth-2mm)/3,colback=white]{logo/amused.jpeg}
        \tcbincludegraphics[boxrule=0.5pt,equal height group=AT,before=,after=\hfill,height=2cm,width=(\linewidth-2mm)/3,colback=white]{logo/haku_dismissive.jpg}
    \end{itemize}

    \tcbsubtitle{À quoi ça sert la physique chimie et qu'est-ce que ça peut m'apporter?}    

    Pas de réponses toutes faites à cette question, il nous faudra le découvrir ensemble!
    \hfill Signatures: .........
    

    \end{titlebox}
\end{document}
