\documentclass[10pt]{article}
\usepackage[table,svgnames,x11names]{xcolor}
\usepackage{geometry}
\usepackage{luacode}
\usepackage{longtable}

\geometry{
	paper=a4paper, % Change to letterpaper for US letter
	top=   0.35cm, % Top margin
	bottom=0.85cm, % Bottom margin
	left= 1cm, % Left margin
	right=1cm, % Right margin
	headheight=14pt, % Header height
	footskip=1.4cm, % Space from the bottom margin to the baseline of the footer
	headsep=1.2cm, % Space from the top margin to the baseline of the header
	%showframe, % Uncomment to show how the type block is set on the page
}



\begin{document}
\thispagestyle{empty}
\directlua{date = "3 janvier 2022"}
\directlua{is_semaine_A = true}

\setlength{\arrayrulewidth}{0.7mm}
\renewcommand{\arraystretch}{1.25}
\arrayrulecolor{red!50!orange!80!black}

imprimer punition 

\begin{luacode*}
    dofile("lua/edt.lua")
	-- my_tex.print(week:display())
\end{luacode*}

{{\Large BONUS} \normalsize : \begin{itemize}
	\item lire l'univers expliqué à mes petits enfant.
	\item demander si fresque du climat ça leur dirait.
\end{itemize}}

\end{document}